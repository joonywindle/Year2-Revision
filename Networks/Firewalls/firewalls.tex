%%%%%%%%%%%%%%%%%%%%%%%%%%%%%%%%%%%%%%%%%
% Beamer Presentation
% LaTeX Template
% Version 1.0 (10/11/12)
%
% This template has been downloaded from:
% http://www.LaTeXTemplates.com
%
% License:
% CC BY-NC-SA 3.0 (http://creativecommons.org/licenses/by-nc-sa/3.0/)
%
%%%%%%%%%%%%%%%%%%%%%%%%%%%%%%%%%%%%%%%%%

%----------------------------------------------------------------------------------------
%	PACKAGES AND THEMES
%----------------------------------------------------------------------------------------

\documentclass{beamer}

\mode<presentation> {

% The Beamer class comes with a number of default slide themes
% which change the colors and layouts of slides. Below this is a list
% of all the themes, uncomment each in turn to see what they look like.

%\usetheme{default}
%\usetheme{AnnArbor}
%\usetheme{Antibes}
%\usetheme{Bergen}
%\usetheme{Berkeley}
%\usetheme{Berlin}
%\usetheme{Boadilla}
%\usetheme{CambridgeUS}
%\usetheme{Copenhagen}
%\usetheme{Darmstadt}
%\usetheme{Dresden}
%\usetheme{Frankfurt}
%\usetheme{Goettingen}
%\usetheme{Hannover}
%\usetheme{Ilmenau}
%\usetheme{JuanLesPins}
%\usetheme{Luebeck}
\usetheme{Madrid}
%\usetheme{Malmoe}
%\usetheme{Marburg}
%\usetheme{Montpellier}
%\usetheme{PaloAlto}
%\usetheme{Pittsburgh}
%\usetheme{Rochester}
%\usetheme{Singapore}
%\usetheme{Szeged}
%\usetheme{Warsaw}

% As well as themes, the Beamer class has a number of color themes
% for any slide theme. Uncomment each of these in turn to see how it
% changes the colors of your current slide theme.

%\usecolortheme{albatross}
%\usecolortheme{beaver}
%\usecolortheme{beetle}
%\usecolortheme{crane}
%\usecolortheme{dolphin}
%\usecolortheme{dove}
%\usecolortheme{fly}
%\usecolortheme{lily}
%\usecolortheme{orchid}
%\usecolortheme{rose}
%\usecolortheme{seagull}
%\usecolortheme{seahorse}
%\usecolortheme{whale}
%\usecolortheme{wolverine}

%\setbeamertemplate{footline} % To remove the footer line in all slides uncomment this line
%\setbeamertemplate{footline}[page number] % To replace the footer line in all slides with a simple slide count uncomment this line

%\setbeamertemplate{navigation symbols}{} % To remove the navigation symbols from the bottom of all slides uncomment this line
}

\usepackage{graphicx} % Allows including images
\usepackage{booktabs} % Allows the use of \toprule, \midrule and \bottomrule in tables
\usepackage{listings}
\usepackage{amsmath}
\usepackage{algpseudocode,algorithm,algorithmicx}

\lstdefinestyle{customjava}{
  breaklines=true,
  frame=L,
  xleftmargin=\parindent,
  language=Java,
  showstringspaces=false,
  basicstyle=\footnotesize\ttfamily,
  keywordstyle=\bfseries\color{green!40!black},
  commentstyle=\itshape\color{gray!40!black},
  identifierstyle=\color{blue},
  stringstyle=\color{orange},
}

%----------------------------------------------------------------------------------------
%	TITLE PAGE
%----------------------------------------------------------------------------------------

\title[Firewalls]{Firewalls} % The short title appears at the bottom of every slide, the full title is only on the title page

\author{Jonathan Windle} % Your name
\institute[UEA] % Your institution as it will appear on the bottom of every slide, may be shorthand to save space
{
University of East Anglia \\ % Your institution for the title page
\medskip
\textit{J.Windle@uea.ac.uk} % Your email address
}
\date{\today} % Date, can be changed to a custom date

\begin{document}

\begin{frame}
\titlepage % Print the title page as the first slide
\end{frame}

\begin{frame}[allowframebreaks]
\frametitle{Overview} % Table of contents slide, comment this block out to remove it
\tableofcontents % Throughout your presentation, if you choose to use \section{} and \subsection{} commands, these will automatically be printed on this slide as an overview of your presentation
\end{frame}

%-----------------------------------------------------------------
\section{Intro}
\begin{frame}
\frametitle{Intro}
\begin{itemize}
\item Essentially if a fire breaks out in one part of the building, it is contained, same concept applies to computing.
\item Firewalls installed on routers.
\item As traffic passes through, the firewall is configured to try and attempt to block malicious content.
\item One way to do this is by packet filetering:
\item A packet filer matches all packets that pass through the router against rules.
\item The rules are based on security policies for the network.
\item If a packet matches a rule, it is either accepted or rejected and the action logged.
\item Default is to reject.
\item Two types:
\begin{itemize}
\item Basic packet filtering
\item Stateful packet filtering.
\end{itemize}
\end{itemize}
\end{frame}
%------------------------------------------------------------------
\section{Basic Packet Filtering}
\begin{frame}
\frametitle{Basic Packet Filtering}
\begin{itemize}
\item Examines each packet in isolation.
\item Does not consider if the packet is a part of a stream or has a relationship with another packet.
\item Stateless operation.
\item Rules often based on a combination of source and destination addresses and protocols.
\item For TCP and UDP, port may also be considered.
\item A firewall could be set to block all traffic to a given port.
\item Different rules may be specified for traffic travelling in each directon:
\begin{itemize}
\item Allow host on private network direct access.
\item Limit access of hosts on the internet to the private network.
\end{itemize}
\item Does not need to dig far into the packet:
\item Most work is done at the first three layers of the OSI protocol.
\end{itemize}
\end{frame}
%-------------------------------------------------------------------
\section{Stateful Packet Filtering}
\begin{frame}
\frametitle{Stateful Packet Filtering}
\begin{itemize}
\item Keeps record of packets already routed.
\item May be sensible to apply different rules to the first packet in a connection to subsequent packets.
\item Keeps a state table of conditions.
\item Drop connnection from state table on:
\begin{itemize}
\item Connection end
\item Time out.
\end{itemize}
\end{itemize}
\end{frame}
%-----------------------------------------------------------------
\begin{frame} 
\Huge{\centerline{The End}}
\end{frame}
\end{document}