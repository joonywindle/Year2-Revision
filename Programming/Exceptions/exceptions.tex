%%%%%%%%%%%%%%%%%%%%%%%%%%%%%%%%%%%%%%%%%
% Beamer Presentation
% LaTeX Template
% Version 1.0 (10/11/12)
%
% This template has been downloaded from:
% http://www.LaTeXTemplates.com
%
% License:
% CC BY-NC-SA 3.0 (http://creativecommons.org/licenses/by-nc-sa/3.0/)
%
%%%%%%%%%%%%%%%%%%%%%%%%%%%%%%%%%%%%%%%%%

%----------------------------------------------------------------------------------------
%	PACKAGES AND THEMES
%----------------------------------------------------------------------------------------

\documentclass{beamer}

\mode<presentation> {

% The Beamer class comes with a number of default slide themes
% which change the colors and layouts of slides. Below this is a list
% of all the themes, uncomment each in turn to see what they look like.

%\usetheme{default}
%\usetheme{AnnArbor}
%\usetheme{Antibes}
%\usetheme{Bergen}
%\usetheme{Berkeley}
%\usetheme{Berlin}
%\usetheme{Boadilla}
%\usetheme{CambridgeUS}
%\usetheme{Copenhagen}
%\usetheme{Darmstadt}
%\usetheme{Dresden}
%\usetheme{Frankfurt}
%\usetheme{Goettingen}
%\usetheme{Hannover}
%\usetheme{Ilmenau}
%\usetheme{JuanLesPins}
%\usetheme{Luebeck}
\usetheme{Madrid}
%\usetheme{Malmoe}
%\usetheme{Marburg}
%\usetheme{Montpellier}
%\usetheme{PaloAlto}
%\usetheme{Pittsburgh}
%\usetheme{Rochester}
%\usetheme{Singapore}
%\usetheme{Szeged}
%\usetheme{Warsaw}

% As well as themes, the Beamer class has a number of color themes
% for any slide theme. Uncomment each of these in turn to see how it
% changes the colors of your current slide theme.

%\usecolortheme{albatross}
%\usecolortheme{beaver}
%\usecolortheme{beetle}
%\usecolortheme{crane}
%\usecolortheme{dolphin}
%\usecolortheme{dove}
%\usecolortheme{fly}
%\usecolortheme{lily}
%\usecolortheme{orchid}
%\usecolortheme{rose}
%\usecolortheme{seagull}
%\usecolortheme{seahorse}
%\usecolortheme{whale}
%\usecolortheme{wolverine}

%\setbeamertemplate{footline} % To remove the footer line in all slides uncomment this line
%\setbeamertemplate{footline}[page number] % To replace the footer line in all slides with a simple slide count uncomment this line

%\setbeamertemplate{navigation symbols}{} % To remove the navigation symbols from the bottom of all slides uncomment this line
}

\usepackage{graphicx} % Allows including images
\usepackage{booktabs} % Allows the use of \toprule, \midrule and \bottomrule in tables
\usepackage{listings}
\usepackage{amsmath}
\usepackage{algpseudocode,algorithm,algorithmicx}


%----------------------------------------------------------------------------------------
%	TITLE PAGE
%----------------------------------------------------------------------------------------

\title[Reflection]{Reflection} % The short title appears at the bottom of every slide, the full title is only on the title page

\author{Jonathan Windle} % Your name
\institute[UEA] % Your institution as it will appear on the bottom of every slide, may be shorthand to save space
{
University of East Anglia \\ % Your institution for the title page
\medskip
\textit{J.Windle@uea.ac.uk} % Your email address
}
\date{\today} % Date, can be changed to a custom date

\begin{document}

\begin{frame}
\titlepage % Print the title page as the first slide
\end{frame}

\begin{frame}[allowframebreaks]
\frametitle{Overview} % Table of contents slide, comment this block out to remove it
\tableofcontents % Throughout your presentation, if you choose to use \section{} and \subsection{} commands, these will automatically be printed on this slide as an overview of your presentation
\end{frame}

%-----------------------------------------------------------------
\section{Intro}
\begin{frame}
\frametitle{Intro}
\begin{itemize}
\item Exceptional events that stop the normal flow of execution.
\item Execution stops at the exact point of exception.
\item {\color{red} Exception object} is created with information about the event.
\item The exception is {\color{green}thrown} down the method stack until it is either {\color{orange} caught} or the program terminates.
\item All built in exceptions inherit from the \texttt{Exception} class.
\end{itemize}
\end{frame}
%-----------------------------------------------------------------
\defverbatim[colored]\try{
\begin{lstlisting}[language=Java,basicstyle=\tiny,keywordstyle=\color{blue}]
try{
// Do stuff that might throw an exception
}
catch(Exception e){ // Thrown exception is stored here
// Corrective action goes here
}
finally {
// This will always execute
}
// Method continues here.
\end{lstlisting}
}
\section{Try-Catch}
\begin{frame}
\frametitle{Try-Catch}
\try
\begin{itemize}
\item Variables declared within the try have scope limitation.
\item Can catch anywhere up the stack.
\item \texttt{finally} always executes, commonly used to clean up, close streams etc.
\item Exceptions can be checked or unchecked.
\end{itemize}
\end{frame}
%-----------------------------------------------------------------
\section{Checked vs Unchecked}
\begin{frame}
\frametitle{Checked vs Unchecked}
\begin{itemize}
\item {\color{green} Checked} makes sure the code that may throw the \texttt{Exception} has to be surrounded by try-catch or has a throws clause. 
\item {\color{red} Unchecked} means that the code may throw an exception, but doesn't have to be enclosed in try-catch or throws. e.g. \texttt{c = a/b} could throw an exception if b is 0, but it doesn't need checks.
\item Exceptions inheriting from \texttt{RunTimeException} are by default {\color{red}unchecked}.
\item Can make any Exception {\color{green} checked} by adding \texttt{\color{purple} throws}.
\end{itemize}
\end{frame}
%------------------------------------------------------------------
\section{Advantages}
\begin{frame}
\frametitle{Advantages}
\begin{itemize}
\item Can seperate Error-Handling code from application code.
\item Can propagate exceptions up the stack and therefore handle them in a suitable place (e.g. GUI).
\item Can use inheritance to group types of exception and thus increase the information to convey.
\end{itemize}
\end{frame}
%-----------------------------------------------------------------
\section{Summary}
\begin{frame}
\frametitle{Summary}
\begin{itemize}
\item Are Exceptional events that disrupt the normal flow of execution.
\item Thrown using \texttt{throws}.
\item Caught in a \texttt{try-catch-finally} block.
\item Can be propagated up the stack and caught at any point.
\item Seperate error-handling code from application code.
\end{itemize}
\end{frame}
%-----------------------------------------------------------------


\begin{frame} 
\Huge{\centerline{The End}}
\end{frame}

\end{document}