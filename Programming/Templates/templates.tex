%%%%%%%%%%%%%%%%%%%%%%%%%%%%%%%%%%%%%%%%%
% Beamer Presentation
% LaTeX Template
% Version 1.0 (10/11/12)
%
% This template has been downloaded from:
% http://www.LaTeXTemplates.com
%
% License:
% CC BY-NC-SA 3.0 (http://creativecommons.org/licenses/by-nc-sa/3.0/)
%
%%%%%%%%%%%%%%%%%%%%%%%%%%%%%%%%%%%%%%%%%

%----------------------------------------------------------------------------------------
%	PACKAGES AND THEMES
%----------------------------------------------------------------------------------------

\documentclass{beamer}

\mode<presentation> {

% The Beamer class comes with a number of default slide themes
% which change the colors and layouts of slides. Below this is a list
% of all the themes, uncomment each in turn to see what they look like.

%\usetheme{default}
%\usetheme{AnnArbor}
%\usetheme{Antibes}
%\usetheme{Bergen}
%\usetheme{Berkeley}
%\usetheme{Berlin}
%\usetheme{Boadilla}
%\usetheme{CambridgeUS}
%\usetheme{Copenhagen}
%\usetheme{Darmstadt}
%\usetheme{Dresden}
%\usetheme{Frankfurt}
%\usetheme{Goettingen}
%\usetheme{Hannover}
%\usetheme{Ilmenau}
%\usetheme{JuanLesPins}
%\usetheme{Luebeck}
\usetheme{Madrid}
%\usetheme{Malmoe}
%\usetheme{Marburg}
%\usetheme{Montpellier}
%\usetheme{PaloAlto}
%\usetheme{Pittsburgh}
%\usetheme{Rochester}
%\usetheme{Singapore}
%\usetheme{Szeged}
%\usetheme{Warsaw}

% As well as themes, the Beamer class has a number of color themes
% for any slide theme. Uncomment each of these in turn to see how it
% changes the colors of your current slide theme.

%\usecolortheme{albatross}
%\usecolortheme{beaver}
%\usecolortheme{beetle}
%\usecolortheme{crane}
%\usecolortheme{dolphin}
%\usecolortheme{dove}
%\usecolortheme{fly}
%\usecolortheme{lily}
%\usecolortheme{orchid}
%\usecolortheme{rose}
%\usecolortheme{seagull}
%\usecolortheme{seahorse}
%\usecolortheme{whale}
%\usecolortheme{wolverine}

%\setbeamertemplate{footline} % To remove the footer line in all slides uncomment this line
%\setbeamertemplate{footline}[page number] % To replace the footer line in all slides with a simple slide count uncomment this line

%\setbeamertemplate{navigation symbols}{} % To remove the navigation symbols from the bottom of all slides uncomment this line
}

\usepackage{graphicx} % Allows including images
\usepackage{booktabs} % Allows the use of \toprule, \midrule and \bottomrule in tables
\usepackage{listings}
\usepackage{amsmath}
\usepackage{algpseudocode,algorithm,algorithmicx}

\lstdefinestyle{customjava}{
  breaklines=true,
  frame=L,
  xleftmargin=\parindent,
  language=Java,
  showstringspaces=false,
  basicstyle=\footnotesize\ttfamily,
  keywordstyle=\bfseries\color{green!40!black},
  commentstyle=\itshape\color{gray!40!black},
  identifierstyle=\color{blue},
  stringstyle=\color{orange},
}

\lstdefinestyle{customcpp}{
  breaklines=true,
  frame=L,
  xleftmargin=\parindent,
  language=C++,
  showstringspaces=false,
  basicstyle=\footnotesize\ttfamily,
  keywordstyle=\bfseries\color{green!40!black},
  commentstyle=\itshape\color{gray!40!black},
  identifierstyle=\color{blue},
  stringstyle=\color{orange},
}
%----------------------------------------------------------------------------------------
%	TITLE PAGE
%----------------------------------------------------------------------------------------

\title[Templates]{Templates} % The short title appears at the bottom of every slide, the full title is only on the title page

\author{Jonathan Windle} % Your name
\institute[UEA] % Your institution as it will appear on the bottom of every slide, may be shorthand to save space
{
University of East Anglia \\ % Your institution for the title page
\medskip
\textit{J.Windle@uea.ac.uk} % Your email address
}
\date{\today} % Date, can be changed to a custom date

\begin{document}

\begin{frame}
\titlepage % Print the title page as the first slide
\end{frame}

\begin{frame}[allowframebreaks]
\frametitle{Overview} % Table of contents slide, comment this block out to remove it
\tableofcontents % Throughout your presentation, if you choose to use \section{} and \subsection{} commands, these will automatically be printed on this slide as an overview of your presentation
\end{frame}

%-------------------------------------------------------------------
\section{Intro}
\begin{frame}
\frametitle{Intro}
\begin{itemize}
\item Code synthesized for required classes by expanding template.
\item Synthesized code is then compiled as well.
\item Advantages: No impact on runtime performance, easy for the compiler to optimise code.
\item Disadvantages: Need to know what synthesized code is needed at compile time. Larger executable.
\end{itemize}
\end{frame}

%-------------------------------------------------------------------
\defverbatim[colored]\temp{
\begin{lstlisting}[style = customcpp]
template <typename T>
void exchange(T& x, T& y){
	T tmp = x;
	x = y;
	y = tmp;
}
\end{lstlisting}
}
\defverbatim[colored]\syn{
\begin{lstlisting}[style = customcpp]
// New identifier synthesized by compiler
void _exchange_int(int& x, int& y) {
	int tmp = x;
	x = y;
	y = tmp;
}
\end{lstlisting}
}
\section{Template Functions}
\begin{frame}
\frametitle{Template Functions}
\begin{itemize}
\item When the function is invoked, the compiler:
\begin{itemize}
\item Deduces the values of the templates parameter.
\item Determines whether it exists already or not.
\item If not, a new instance is generated by expanding the template.
\item Compiles the new instance of the template
\end{itemize}
\temp
\item Exchanging two integers would result a synthesized function:
\syn
\end{itemize}
\end{frame}
%-----------------------------------------------------------------
\section{Type Safety}
\begin{frame}
\frametitle{Type Safety}
\begin{itemize}
\item The compiler performs type checking on synthesized code.
\item If trying to swap and integer and a double, there will be an error at compile time.
\item Opportunities for problems to be detected:
\begin{itemize}
\item When the template itself is compiled
\item When the compiled detects a use of the template.
\item When the instance of the template is compiled.
\end{itemize}
\end{itemize}
\end{frame}
%------------------------------------------------------------------
\defverbatim[colored]\exp{
\begin{lstlisting}[style = customcpp]
template <typname R, typename X, typename Y>
R ratio(X& x, Y& y) {
	return x/static_cast<R>(y);
}

int main(int argc, char* argv[]) {
	int i = 87;
	short s = 42;

// R set explicitly, X & Y deduced implicitly.
	cout << ratio<int>(i,s) << endl; // 2
	
	cout << ratio<double>(i,s) << endl;// 2.07134
	
	return 0;
}
\end{lstlisting}
}
\section{Explicit Template Arguments}
\begin{frame}
\frametitle{Explicit Template Arguments}
\begin{itemize}
\item Sometimes types cannot be deduced from the invocation.
\item E.g Return type of a function:
\exp
\vspace{-0.5cm}
\item Deduced template parameters must be right most in the list.
\end{itemize}
\end{frame}
%----------------------------------------------------------------
\defverbatim[colored]\non{
\begin{lstlisting}[style = customcpp]
template <int N>
void repeat(const String& s) {
	for (int i = 0; i < N; i++) {
		cout << s << endl;
	}
}

int main(int argc, char* argv[]) {
// Template argument given explicitly
	repeat<3>("Hello world");
	
	int n = 4;
// Local variable cannot be used as a non-type argument
	repeat<n>("Won't work");
	return 0;
}
\end{lstlisting}
}
\section{Nontype Template Parameters}
\begin{frame}
\frametitle{Nontype Template Parameters}
\begin{itemize}
\item They do not have to be typenames:
\non
\item Template instantiated at compile time, so arguments must be known at compile time.
\end{itemize}
\end{frame}
%-----------------------------------------------------------------
\defverbatim[colored]\arr{
\begin{lstlisting}[style = customcpp]
template <typename T, int N>
int arrayLength(const T(&array)[N]) { // N part of array type
	return N;
}
\end{lstlisting}
}
\section{Array length can be deduced}
\begin{frame}
\frametitle{Array length can be deduced}
\arr
\begin{itemize}
\item This works because the compiler deduces that the parameter must be type T and size N.
\end{itemize}
\end{frame}

%------------------------------------------------------------------
\defverbatim[colored]\class{
\begin{lstlisting}[style = customcpp]
template <typename T> class Array {
private:
	T* array;
	int N;
public:
// Method for overloaded operator declared
	T& operator[](int i);
	Array(const int size){
		array = new T[size];
		N = size;
	}
};
// Generic so must be defined as a template.
template <typename T>
T& Array<T>::operator[](int i) {
	return array[i];
}
\end{lstlisting}
}
\section{Class Templates}
\begin{frame}
\frametitle{Class Templates}
\begin{itemize}
\item Used in a very similar way to methods.
\item Example definition:
\class
\end{itemize}
\end{frame}

%-----------------------------------------------------------------
\defverbatim[colored]\for{
\begin{lstlisting}[style = customcpp]
template <typename T> class Array {
private:
	T* array;
	int N;
public:
	T* begin() { return array; }
	
	T* end() { return array+N; }
	
	int length() const { return N; }
	
	T& operator[](int i) { return array[i]; }
	
	Array(const int size) {
		array = new T[size];
		N = size;
	}
}
\end{lstlisting}
}
\section{Ranged for loop}
\begin{frame}[allowframebreaks]
\frametitle{Ranged for loop}
\begin{itemize}
\item To work a collection must have:
\begin{itemize}
\item A \texttt{begin()} method that returns the iterator to the first element.
\item An \texttt{end()} method that returns an iterator for the element just past the end of the collection.
\end{itemize}
\item Iterator is an object that:
\begin{itemize}
\item Overloads the * operator to return a reference to an element.
\item Overloads the ++ operator to move the iterator onto the next element.
\item Overloads the relational operators, e.g. ==, !=, <, > etc...
\end{itemize}
\item Pointers behave like iterators.
\begin{itemize}
\item We can use a pointer for an iterator
\item Example isn't very safe...
\end{itemize}
\end{itemize}
\for
\end{frame}
%------------------------------------------------------------------

\begin{frame} 
\Huge{\centerline{The End}}
\end{frame}

\end{document}