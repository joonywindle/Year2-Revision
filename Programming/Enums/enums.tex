%%%%%%%%%%%%%%%%%%%%%%%%%%%%%%%%%%%%%%%%%
% Beamer Presentation
% LaTeX Template
% Version 1.0 (10/11/12)
%
% This template has been downloaded from:
% http://www.LaTeXTemplates.com
%
% License:
% CC BY-NC-SA 3.0 (http://creativecommons.org/licenses/by-nc-sa/3.0/)
%
%%%%%%%%%%%%%%%%%%%%%%%%%%%%%%%%%%%%%%%%%

%----------------------------------------------------------------------------------------
%	PACKAGES AND THEMES
%----------------------------------------------------------------------------------------

\documentclass{beamer}

\mode<presentation> {

% The Beamer class comes with a number of default slide themes
% which change the colors and layouts of slides. Below this is a list
% of all the themes, uncomment each in turn to see what they look like.

%\usetheme{default}
%\usetheme{AnnArbor}
%\usetheme{Antibes}
%\usetheme{Bergen}
%\usetheme{Berkeley}
%\usetheme{Berlin}
%\usetheme{Boadilla}
%\usetheme{CambridgeUS}
%\usetheme{Copenhagen}
%\usetheme{Darmstadt}
%\usetheme{Dresden}
%\usetheme{Frankfurt}
%\usetheme{Goettingen}
%\usetheme{Hannover}
%\usetheme{Ilmenau}
%\usetheme{JuanLesPins}
%\usetheme{Luebeck}
\usetheme{Madrid}
%\usetheme{Malmoe}
%\usetheme{Marburg}
%\usetheme{Montpellier}
%\usetheme{PaloAlto}
%\usetheme{Pittsburgh}
%\usetheme{Rochester}
%\usetheme{Singapore}
%\usetheme{Szeged}
%\usetheme{Warsaw}

% As well as themes, the Beamer class has a number of color themes
% for any slide theme. Uncomment each of these in turn to see how it
% changes the colors of your current slide theme.

%\usecolortheme{albatross}
%\usecolortheme{beaver}
%\usecolortheme{beetle}
%\usecolortheme{crane}
%\usecolortheme{dolphin}
%\usecolortheme{dove}
%\usecolortheme{fly}
%\usecolortheme{lily}
%\usecolortheme{orchid}
%\usecolortheme{rose}
%\usecolortheme{seagull}
%\usecolortheme{seahorse}
%\usecolortheme{whale}
%\usecolortheme{wolverine}

%\setbeamertemplate{footline} % To remove the footer line in all slides uncomment this line
%\setbeamertemplate{footline}[page number] % To replace the footer line in all slides with a simple slide count uncomment this line

%\setbeamertemplate{navigation symbols}{} % To remove the navigation symbols from the bottom of all slides uncomment this line
}

\usepackage{graphicx} % Allows including images
\usepackage{booktabs} % Allows the use of \toprule, \midrule and \bottomrule in tables
\usepackage{listings}

%----------------------------------------------------------------------------------------
%	TITLE PAGE
%----------------------------------------------------------------------------------------

\title[Nested Classes]{Nested Classes} % The short title appears at the bottom of every slide, the full title is only on the title page

\author{Jonathan Windle} % Your name
\institute[UEA] % Your institution as it will appear on the bottom of every slide, may be shorthand to save space
{
University of East Anglia \\ % Your institution for the title page
\medskip
\textit{J.Windle@uea.ac.uk} % Your email address
}
\date{\today} % Date, can be changed to a custom date

\begin{document}

\begin{frame}
\titlepage % Print the title page as the first slide
\end{frame}

\begin{frame}[allowframebreaks]
\frametitle{Overview} % Table of contents slide, comment this block out to remove it
\tableofcontents % Throughout your presentation, if you choose to use \section{} and \subsection{} commands, these will automatically be printed on this slide as an overview of your presentation
\end{frame}
%-------------------------------------------------------------
\section{Intro}
\begin{frame}
\frametitle{Intro}
\begin{itemize}
\item \texttt{Enum} types are a means of modelling a variable with a discrete, finite number of values.
\item They provide type safety, modularity, clarity and flexibility.
\item They are instances of {\color{red} Anonymous inner classes}.
\item In C++ they are just integers, and can be treated as such, in Java, they are in fact each an instantiation of an anonymous inner class.
\end{itemize}
\end{frame}
%----------------------------------------------------------------
\section{Enum Extra Features}
\begin{frame}
\frametitle{Enum Extra Features}
\begin{itemize}
\item Because it's a class, it has built in methods such as:
\begin{itemize}
\item \texttt{values()}: Returns the possible values as an array.
\item \texttt{ordinal()}: Returns the rank of the enum as an integer.
\end{itemize}
\item \texttt{values()} can be used to iterate over enums using \texttt{for each}.
\end{itemize}
\end{frame}
%------------------------------------------------------------------
\defverbatim[colored]\advanced{
\begin{lstlisting}[language=Java,basicstyle=\scriptsize,keywordstyle=\color{blue}]
enum Grade{
// Create instances of Grade:
FIRST(70), TWO_ONE(60), TWO_TWO(50), THIRD(40), FAIL(0);
// Define class data:
	final int boundary; 
	Grade(int x){
		boundary=x;
	}
	public double getBoundary(){return boundary;}
}

Grade g=Grade.FIRST;
System.out.print("g = "+g.boundary);
\end{lstlisting}
}
\section{Advanced usage of Enums}
\begin{frame}
\frametitle{Advanced usage of Enums}
\begin{itemize}
\item Can essentially give an enum a constructor and treat them as a class, e.g:
\advanced
\item They can be mutable (Can change data) generally don't use enums in this case, but can be useful as they can model the singleton pattern.
\end{itemize}
\end{frame}
%--------------------------------------------------------------
\section{Summary}
\begin{frame}
\frametitle{Summary}
\begin{itemize}
\item Ensure {\color{green} Type safety}, errors detected at compile time.
\item {\color{red} Clarity}, makes code easier to understand.
\item {\color{orange} Modularity}, add values to existing enum without changing any code that uses the enum type.
\item {\color{purple} Flexibility}, built in function are handy.
\end{itemize}
\end{frame}
%----------------------------------------------------------------
\defverbatim[colored]\ellipseEx{
\begin{lstlisting}[language=Java,basicstyle=\scriptsize,keywordstyle=\color{blue}]
public static double average(double...numbers){
	double s=0;
	for(int i=0;i<numbers.length;i++)
		s+=numbers[i];
	return s/numbers.length;
}

public static void main(String[] args){
	double d1=100,d2=33,d3=333;
	double mean;
  	mean=average(d1);
  	mean=average(d1,d2);
  	mean=average(d1,d2,d3);

}
\end{lstlisting}
}
\section{Ellipsis Operator}
\begin{frame}
\frametitle{Ellipsis Operator}
\begin{itemize}
\item Allows user to enter a variable number of arguments of a type.
\item You {\color{red} cannot} assign default values to parameters (You can in C++).
\item Takes parameters in as an array.
\ellipseEx
\end{itemize}
\end{frame}
%-----------------------------------------------------------------
\begin{frame}
\Huge{\centerline{The End}}
\end{frame}

\end{document}