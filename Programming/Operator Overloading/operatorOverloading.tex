%%%%%%%%%%%%%%%%%%%%%%%%%%%%%%%%%%%%%%%%%
% Beamer Presentation
% LaTeX Template
% Version 1.0 (10/11/12)
%
% This template has been downloaded from:
% http://www.LaTeXTemplates.com
%
% License:
% CC BY-NC-SA 3.0 (http://creativecommons.org/licenses/by-nc-sa/3.0/)
%
%%%%%%%%%%%%%%%%%%%%%%%%%%%%%%%%%%%%%%%%%

%----------------------------------------------------------------------------------------
%	PACKAGES AND THEMES
%----------------------------------------------------------------------------------------

\documentclass{beamer}

\mode<presentation> {

% The Beamer class comes with a number of default slide themes
% which change the colors and layouts of slides. Below this is a list
% of all the themes, uncomment each in turn to see what they look like.

%\usetheme{default}
%\usetheme{AnnArbor}
%\usetheme{Antibes}
%\usetheme{Bergen}
%\usetheme{Berkeley}
%\usetheme{Berlin}
%\usetheme{Boadilla}
%\usetheme{CambridgeUS}
%\usetheme{Copenhagen}
%\usetheme{Darmstadt}
%\usetheme{Dresden}
%\usetheme{Frankfurt}
%\usetheme{Goettingen}
%\usetheme{Hannover}
%\usetheme{Ilmenau}
%\usetheme{JuanLesPins}
%\usetheme{Luebeck}
\usetheme{Madrid}
%\usetheme{Malmoe}
%\usetheme{Marburg}
%\usetheme{Montpellier}
%\usetheme{PaloAlto}
%\usetheme{Pittsburgh}
%\usetheme{Rochester}
%\usetheme{Singapore}
%\usetheme{Szeged}
%\usetheme{Warsaw}

% As well as themes, the Beamer class has a number of color themes
% for any slide theme. Uncomment each of these in turn to see how it
% changes the colors of your current slide theme.

%\usecolortheme{albatross}
%\usecolortheme{beaver}
%\usecolortheme{beetle}
%\usecolortheme{crane}
%\usecolortheme{dolphin}
%\usecolortheme{dove}
%\usecolortheme{fly}
%\usecolortheme{lily}
%\usecolortheme{orchid}
%\usecolortheme{rose}
%\usecolortheme{seagull}
%\usecolortheme{seahorse}
%\usecolortheme{whale}
%\usecolortheme{wolverine}

%\setbeamertemplate{footline} % To remove the footer line in all slides uncomment this line
%\setbeamertemplate{footline}[page number] % To replace the footer line in all slides with a simple slide count uncomment this line

%\setbeamertemplate{navigation symbols}{} % To remove the navigation symbols from the bottom of all slides uncomment this line
}

\usepackage{graphicx} % Allows including images
\usepackage{booktabs} % Allows the use of \toprule, \midrule and \bottomrule in tables
\usepackage{listings}
\usepackage{amsmath}
\usepackage{algpseudocode,algorithm,algorithmicx}

\lstdefinestyle{customjava}{
  breaklines=true,
  frame=L,
  xleftmargin=\parindent,
  language=Java,
  showstringspaces=false,
  basicstyle=\footnotesize\ttfamily,
  keywordstyle=\bfseries\color{green!40!black},
  commentstyle=\itshape\color{gray!40!black},
  identifierstyle=\color{blue},
  stringstyle=\color{orange},
}

\lstdefinestyle{customcpp}{
  breaklines=true,
  frame=L,
  xleftmargin=\parindent,
  language=C++,
  showstringspaces=false,
  basicstyle=\footnotesize\ttfamily,
  keywordstyle=\bfseries\color{green!40!black},
  commentstyle=\itshape\color{gray!40!black},
  identifierstyle=\color{blue},
  stringstyle=\color{orange},
}
%----------------------------------------------------------------------------------------
%	TITLE PAGE
%----------------------------------------------------------------------------------------

\title[Operator Overloading]{Operator Overloading} % The short title appears at the bottom of every slide, the full title is only on the title page

\author{Jonathan Windle} % Your name
\institute[UEA] % Your institution as it will appear on the bottom of every slide, may be shorthand to save space
{
University of East Anglia \\ % Your institution for the title page
\medskip
\textit{J.Windle@uea.ac.uk} % Your email address
}
\date{\today} % Date, can be changed to a custom date

\begin{document}

\begin{frame}
\titlepage % Print the title page as the first slide
\end{frame}

\begin{frame}[allowframebreaks]
\frametitle{Overview} % Table of contents slide, comment this block out to remove it
\tableofcontents % Throughout your presentation, if you choose to use \section{} and \subsection{} commands, these will automatically be printed on this slide as an overview of your presentation
\end{frame}

%-------------------------------------------------------------------
\section{Intro}
\begin{frame}
\frametitle{Intro}
\begin{itemize}
\item Redefine function of operators for specific types.
\item No more complicated than writing function version.
\item Create methods with specific names.
\item Compiler uses operand types to determine correct method/function.
\end{itemize}
\end{frame}

%------------------------------------------------------------------
\defverbatim[colored]\fra{
\begin{lstlisting}[style = customcpp]
class Fraction
{
private:
	int numerator, denominator;
public:
// Begin with operator, redefine behaviour of *
// Redefine behaviour for two Fraction operands
	Fraction operator*(Fraction& rf) {
		Fraction result(0,1);
		
		result.numerator = numerator*rf.numerator;
		result.denominator = denominator*rf.denominator;
		
		return result;	
	}
}
\end{lstlisting}
}
\section{Operator overloading}
\subsection{Method Example}
\begin{frame}
\frametitle{Method Example}
\fra
\vspace{-0.5cm}
\begin{itemize}
\item This is a {\color{red}method} version of overloading.
\item Is a {\color{red} method} because it belongs to a class and takes one argument as the right side operand.
\end{itemize}
\end{frame}

%------------------------------------------------------------------------
\defverbatim[colored]\fraf{
\begin{lstlisting}[style = customcpp,basicstyle=\tiny]
class Fraction
{
private:
	int numerator, denominator;
public:
// Accessors don't break encapsulation
// inline to avoid method call overhead
	inline int getNumerator() const {
		return numerator;
	}
	inline int getDenominator() const {
		return denominator
	}
};

// Minimise amount of code that directly depends on internal implementation.

inline Fraction operator*(const Fraction& lf,const Fraction& rf) {
	
	int numerator, denominator;
	
	numerator = lf.numerator*rf.numerator;
	denominator = lf.denominator*rf.denominator;
	
	return Fraction(numerator,denominator);	
}
\end{lstlisting}
}
\subsection{Function Example}
\begin{frame}
\frametitle{Function Example}
\fraf
\vspace{-0.75cm}
\begin{itemize}
\item Passing by reference is more efficient, but provide guarantee object can not be modified by method by using \texttt{const}. Limits lines of code that can modify a variable, makes debugging/maintenance easier.

\end{itemize}

\end{frame}
%-----------------------------------------------------------------------
\subsubsection{More on operators}
\begin{frame}
\frametitle{More on operators}
\begin{itemize}
\item Can overload operators for different types, e.g.\\ \texttt{Fraction {\color{blue}operator}*({\color{blue}int} lh, {\color{blue}const} Fraction\& rh)}.
\item Allows integer multiplication with a Fraction. This only allows \texttt{i*frac} and not \texttt{frac*i}.
\item If implemented one way, best to have both to support \texttt{frac*i} too.
\item Another way is to use conversions...
\end{itemize}
\end{frame}
%-------------------------------------------------------------------
\defverbatim[colored]\con{
\begin{lstlisting}[style = customcpp]
Fraction(int n) {
	numerator = n;
	denominator = 1;
}
\end{lstlisting}
}
\subsection{Conversions}
\begin{frame}
\frametitle{Conversions}
\begin{itemize}
\item Constructors define conversions:
\item Add constructor that takes a single int:
\con
\item Use the single operator definition of \\
\texttt{Fraction {\color{blue}operator}*({\color{blue}const} Fraction\& lh, {\color{blue}const} Fraction\& rh)}.
\item If this is in place, the compiler {\color{green} implicitly} casts the integer to a \texttt{Fraction} when either \texttt{i*frac} or \texttt{frac*i} is called.
\end{itemize}
\end{frame}

%----------------------------------------------------------------
\defverbatim[colored]\dob{
\begin{lstlisting}[style = customcpp]
operator double() const {
	return numerator/static_cast<double>(denominator);
}
\end{lstlisting}
}
\defverbatim[colored]\div{
\begin{lstlisting}[style = customcpp]
// frac implicitly converted to double.
double d = frac/3.0;
\end{lstlisting}
}
\section{Conversions - Extended}
\begin{frame}
\frametitle{Conversions - Extended}
\begin{itemize}
\item Can convert to other types using:
\dob
\vspace{-0.25cm}
\item Can then easily convert to \texttt{double} and do things like:
\div
\vspace{-0.25cm}
\item ISSUE! Now two ways of implementing \texttt{i*frac}:
\begin{itemize}
\item Convert \texttt{i} into a fraction and use overload as before...
\item Convert \texttt{frac} into a double and use build in *.
\item Ambiguity like this can stop it compiling.
\end{itemize}
\item Can stop this issue by using the \texttt{\color{blue} explicit} keyword in front of the constructor. This stops implicit conversions from taking place.
\item Can still be used as an explicit conversion using \texttt{{\color{blue}static\_cast}<Fraction>(4);}
\end{itemize}
\end{frame}
%-----------------------------------------------------------------
\defverbatim[colored]\rele{
\begin{lstlisting}[style = customcpp, basicstyle = \scriptsize]
inline bool operator == (const Fraction& f, const Fraction& g) {
	return f.getNumerator()*g.getDenominator() == g.getNumerator()*g.getDenominator();
}
\end{lstlisting}
}
\defverbatim[colored]\reln{
\begin{lstlisting}[style = customcpp, basicstyle = \scriptsize]
inline bool operator != (const Fraction& f, const Fraction& g) {
	return !(f==g);
}
\end{lstlisting}
}
\section{Relational Operators}
\begin{frame}
\frametitle{Relational Operators}
\begin{itemize}
\item Often good to implement relational operators e.g:
\rele
\item With one defined, often easier to define other operators with regards to the original.
\reln
\end{itemize}
\end{frame}
%-----------------------------------------------------------------
\defverbatim[colored]\ostr{
\begin{lstlisting}[style = customcpp, basicstyle = \scriptsize]
inline ostream& operator << (ostream& str, const Fraction& f) {
	return str << f.getNumerator() << "/" << f.getDenominator();
}
\end{lstlisting}
}
\section{Overloading Streams}
\subsection{Ostream}
\begin{frame}
\frametitle{Overloading OStreams}
\ostr
\begin{itemize}
\item Returns a reference to an \texttt{ostream}
\item Takes in an \texttt{ostream} but not as const as it changes.
\item All output streams are subclasses of \texttt{ostream}.
\item Allows output to any output stream.
\end{itemize}
\end{frame}
%-------------------------------------------------------------------
\defverbatim[colored]\istr{
\begin{lstlisting}[style = customcpp, basicstyle = \scriptsize]
inline istream& operator >> (istream& str, Fraction& f) {
	char c;
	int numerator,denominator;
	if(str >> numerator >> c >> denominator) {
		if(c == '/')
			f = Fraction(numerator,denominator);
		else
			str.clear(ios_base::failbit);
	}
	return str; 
}
\end{lstlisting}
}
\subsection{IStream}
\begin{frame}
\frametitle{Overloading IStreams}
\istr 
\begin{itemize}
\item Overload \texttt{istreams} for stream based input.
\item Takes in a reference to the object to alter, in this case \texttt{Fraction f}. 
\item Using \texttt{ios\_base::failbit} makes the stream handle errors gracefully.
\end{itemize}
\end{frame}

\subsubsection{Stream Status Flags}
\begin{frame}
\frametitle{Strewm Status Flags}
\begin{itemize}
\item Streams have condition states to which evaluate to true if successful.
\item Define symbolic constants for each bit:
\begin{itemize}
\item \textit{stream}::badbit - unrecoverable error.
\item \textit{stream}::failbit - recoverable error.
\item \textit{stream}::eofbit - stream has reach the end of file.
\item \textit{stream}::goodbit - stream operated correctly.
\end{itemize}
\item Can enquire about states e.g. \texttt{s.bad()} - returns true if fail or bad bits set.
\item can modify using \texttt{s.clear(flags)} to reset all states apart from those specified.
\item Can set specific status bits using \texttt{s.setstate(flags)}.
\end{itemize}
\end{frame}
%------------------------------------------------------------------
\defverbatim[colored]\fout{
\begin{lstlisting}[style = customcpp, basicstyle = \scriptsize]
ofstream os("file.txt", ofstream::out);
if (os) {
	for(int i = 1; i <= 12; i++){
		for(int j = 1; j <= 12; j++) {
			os << setw(2) << i*j << " ";
		}
		os << endl;
	}
	os.close();
}
else {
	cerr << "Error" << endl;
}
\end{lstlisting}
}
\section{Files}
\subsection{Output}
\begin{frame}
\frametitle{Output file}
\fout
\end{frame}

%-------------------------------------------------------------------
\defverbatim[colored]\fin{
\begin{lstlisting}[style = customcpp, basicstyle = \scriptsize]
ifstream is("file.txt", ifstream::in);
if (is) {
	for(int i = 1; i <= 12; i++){
		for(int j = 1; j <= 12; j++) {
			int n;
			is >> n;
			cout << setw(3) << n << " ";
		}
		cout << endl;
	}
	is.close();
}
else {
	cerr << "Error" << endl;
}
\end{lstlisting}
}
\subsection{Input}
\begin{frame}
\frametitle{Input file}
\fin
\end{frame}
%------------------------------------------------------------------


\begin{frame} 
\Huge{\centerline{The End}}
\end{frame}

\end{document}