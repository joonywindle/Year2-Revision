%%%%%%%%%%%%%%%%%%%%%%%%%%%%%%%%%%%%%%%%%
% Beamer Presentation
% LaTeX Template
% Version 1.0 (10/11/12)
%
% This template has been downloaded from:
% http://www.LaTeXTemplates.com
%
% License:
% CC BY-NC-SA 3.0 (http://creativecommons.org/licenses/by-nc-sa/3.0/)
%
%%%%%%%%%%%%%%%%%%%%%%%%%%%%%%%%%%%%%%%%%

%----------------------------------------------------------------------------------------
%	PACKAGES AND THEMES
%----------------------------------------------------------------------------------------

\documentclass{beamer}

\mode<presentation> {

% The Beamer class comes with a number of default slide themes
% which change the colors and layouts of slides. Below this is a list
% of all the themes, uncomment each in turn to see what they look like.

%\usetheme{default}
%\usetheme{AnnArbor}
%\usetheme{Antibes}
%\usetheme{Bergen}
%\usetheme{Berkeley}
%\usetheme{Berlin}
%\usetheme{Boadilla}
%\usetheme{CambridgeUS}
%\usetheme{Copenhagen}
%\usetheme{Darmstadt}
%\usetheme{Dresden}
%\usetheme{Frankfurt}
%\usetheme{Goettingen}
%\usetheme{Hannover}
%\usetheme{Ilmenau}
%\usetheme{JuanLesPins}
%\usetheme{Luebeck}
\usetheme{Madrid}
%\usetheme{Malmoe}
%\usetheme{Marburg}
%\usetheme{Montpellier}
%\usetheme{PaloAlto}
%\usetheme{Pittsburgh}
%\usetheme{Rochester}
%\usetheme{Singapore}
%\usetheme{Szeged}
%\usetheme{Warsaw}

% As well as themes, the Beamer class has a number of color themes
% for any slide theme. Uncomment each of these in turn to see how it
% changes the colors of your current slide theme.

%\usecolortheme{albatross}
%\usecolortheme{beaver}
%\usecolortheme{beetle}
%\usecolortheme{crane}
%\usecolortheme{dolphin}
%\usecolortheme{dove}
%\usecolortheme{fly}
%\usecolortheme{lily}
%\usecolortheme{orchid}
%\usecolortheme{rose}
%\usecolortheme{seagull}
%\usecolortheme{seahorse}
%\usecolortheme{whale}
%\usecolortheme{wolverine}

%\setbeamertemplate{footline} % To remove the footer line in all slides uncomment this line
%\setbeamertemplate{footline}[page number] % To replace the footer line in all slides with a simple slide count uncomment this line

%\setbeamertemplate{navigation symbols}{} % To remove the navigation symbols from the bottom of all slides uncomment this line
}

\usepackage{graphicx} % Allows including images
\usepackage{booktabs} % Allows the use of \toprule, \midrule and \bottomrule in tables
\usepackage{listings}
\usepackage{amsmath}
\usepackage{algpseudocode,algorithm,algorithmicx}

\lstdefinestyle{customjava}{
  breaklines=true,
  frame=L,
  xleftmargin=\parindent,
  language=Java,
  showstringspaces=false,
  basicstyle=\footnotesize\ttfamily,
  keywordstyle=\bfseries\color{green!40!black},
  commentstyle=\itshape\color{gray!40!black},
  identifierstyle=\color{blue},
  stringstyle=\color{orange},
}

\lstdefinestyle{customcpp}{
  breaklines=true,
  frame=L,
  xleftmargin=\parindent,
  language=C++,
  showstringspaces=false,
  basicstyle=\footnotesize\ttfamily,
  keywordstyle=\bfseries\color{green!40!black},
  commentstyle=\itshape\color{gray!40!black},
  identifierstyle=\color{blue},
  stringstyle=\color{orange},
}
%----------------------------------------------------------------------------------------
%	TITLE PAGE
%----------------------------------------------------------------------------------------

\title[Inheritance]{Inheritance} % The short title appears at the bottom of every slide, the full title is only on the title page

\author{Jonathan Windle} % Your name
\institute[UEA] % Your institution as it will appear on the bottom of every slide, may be shorthand to save space
{
University of East Anglia \\ % Your institution for the title page
\medskip
\textit{J.Windle@uea.ac.uk} % Your email address
}
\date{\today} % Date, can be changed to a custom date

\begin{document}

\begin{frame}
\titlepage % Print the title page as the first slide
\end{frame}

\begin{frame}[allowframebreaks]
\frametitle{Overview} % Table of contents slide, comment this block out to remove it
\tableofcontents % Throughout your presentation, if you choose to use \section{} and \subsection{} commands, these will automatically be printed on this slide as an overview of your presentation
\end{frame}

%-------------------------------------------------------------------
\section{Intro}
\begin{frame}
\frametitle{Intro}
\begin{itemize}
\item Advantages:
\begin{itemize}
\item Encourages code use
\item Encourages modularity, data hiding and encapsulation
\item Makes programs maintainable and easily extendable
\end{itemize}
\item Disadvantages:
\begin{itemize}
\item Can lead to overly complicated implementations.
\item Introduces computational overheads.
\end{itemize}
\end{itemize}
\end{frame}
%-----------------------------------------------------------------
\defverbatim[colored]\abs{
\begin{lstlisting}[style = customcpp]
class Speaker
{
private:
// Inherited, but not visible in sub-class
	string utterance;
public:
// Can be overridden, inherited and visible in sub-class	
	virtual string getUtterance() {
		return utterance;	
	}
// Pure virtual, must be overridden before instantiation
	virtual void speak() = 0;

	Speaker(string utterance) {
		this->utterance = utterance;
	}
};
\end{lstlisting}
}
\section{Inheritance}
\subsection{Base class}
\begin{frame}
\frametitle{Inheritance}
\begin{itemize}
\item Abstract Base class:
\abs
\end{itemize}
\end{frame}

%----------------------------------------------------------------
\defverbatim[colored]\sub{
\begin{lstlisting}[style = customcpp, basicstyle = \tiny]
class Philosopher : public Speaker
{
public:
// Override pure virtual method, can now instantiate
	void speak() {
		cout << "My philosophy is..." << getUtterance() << endl;
	}
	
// Add new methods to inherited behaviour 
	inline void philosophosise() {
		cout << "But what do I mean..." << getUtterance() << endl;
	}
// Over-ride inherited virtual method, these cannot be inline
	string getUtterance() {
		return "Woof-woof";
	}

// Call super-class constructor
	Philosopher(string philosophy) : Speaker(philosophy)
	{
	}
};
\end{lstlisting}
}
\subsection{Sub Class}
\begin{frame}
\frametitle{Sub Class}
\begin{itemize}
\item Concrete Sub Class:
\end{itemize}
\sub
\end{frame}

%------------------------------------------------------------------
\defverbatim[colored]\poly{
\begin{lstlisting}[style = customcpp]
int main(int argc, char* argv[]) {
	Philosopher descartes("I think, therefore I am");
	Dog rover;
	Speaker* speaker = &descartes;
	
// Uses speak in Philosopher class
	speaker->speak();
	speaker = &rover;
	
// Uses speak in Dog class
	speaker->speak();
	Speaker &orater = descartes;
	
// Pointers and references can be polymorphic
	orater.speak();
	return 0;
}
\end{lstlisting}
}
\section{Polymorphic pointers and references}
\begin{frame}
\frametitle{Polymorphic pointers and references}
\begin{itemize}
\item Superclass pointers can point to sub-class objects.
\item Superclass references can reference sub-class objects.
\item Can only access methods present in the super-class.
\poly
\end{itemize}
\end{frame}
%------------------------------------------------------------------
\section{Summary}
\begin{frame}
\frametitle{Summary}
\begin{itemize}
\item Differences between Java and C++:
\begin{itemize}
\item Java methods can be over-riden unless declared \texttt{final}.
\item C++ methods cannot be over-riden unless declared \texttt{virtual}.
\item In C++ inheritance can be \texttt{public} or \texttt{private}.
\begin{itemize}
\item \texttt{public}: users of sub-class can use public members inherited from base class.
\item \texttt{private}: users of sub-class cannot access public members inherited from base class.
\end{itemize}
\item Java inheritance is always \texttt{public}
\item Java has implicit base class \texttt{Object}, C++ doesn't.
\end{itemize}
\item Both C++ and Java provide dynamic binding.
\begin{itemize}
\item Method invoked depends on the type of object, not the type of reference/pointer.
\item The methods available, depend on the type of reference/pointer.
\end{itemize}
\end{itemize}
\end{frame}

%------------------------------------------------------------------
\section{Multiple Inheritance}
\begin{frame}
\frametitle{Multiple Inheritance}
\end{frame}
%------------------------------------------------------------------

\begin{frame} 
\Huge{\centerline{The End}}
\end{frame}

\end{document}