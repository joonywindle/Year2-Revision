%%%%%%%%%%%%%%%%%%%%%%%%%%%%%%%%%%%%%%%%%
% Beamer Presentation
% LaTeX Template
% Version 1.0 (10/11/12)
%
% This template has been downloaded from:
% http://www.LaTeXTemplates.com
%
% License:
% CC BY-NC-SA 3.0 (http://creativecommons.org/licenses/by-nc-sa/3.0/)
%
%%%%%%%%%%%%%%%%%%%%%%%%%%%%%%%%%%%%%%%%%

%----------------------------------------------------------------------------------------
%	PACKAGES AND THEMES
%----------------------------------------------------------------------------------------

\documentclass{beamer}

\mode<presentation> {

% The Beamer class comes with a number of default slide themes
% which change the colors and layouts of slides. Below this is a list
% of all the themes, uncomment each in turn to see what they look like.

% \usetheme{default}
%\usetheme{AnnArbor}
%\usetheme{Antibes}
% \usetheme{Bergen}
%\usetheme{Berkeley}
%\usetheme{Berlin}
%\usetheme{Boadilla}
% \usetheme{CambridgeUS}
%\usetheme{Copenhagen}
%\usetheme{Darmstadt}
%\usetheme{Dresden}
% \usetheme{Frankfurt}
%\usetheme{Goettingen}
%\usetheme{Hannover}
%\usetheme{Ilmenau}
%\usetheme{JuanLesPins}
%\usetheme{Luebeck}
\usetheme{Madrid}
%\usetheme{Malmoe}
%\usetheme{Marburg}
%\usetheme{Montpellier}
%\usetheme{PaloAlto}
%\usetheme{Pittsburgh}
%\usetheme{Rochester}
%\usetheme{Singapore}
%\usetheme{Szeged}
%\usetheme{Warsaw}

% As well as themes, the Beamer class has a number of color themes
% for any slide theme. Uncomment each of these in turn to see how it
% changes the colors of your current slide theme.

%\usecolortheme{albatross}
%\usecolortheme{beaver}
%\usecolortheme{beetle}
%\usecolortheme{crane}
%\usecolortheme{dolphin}
%\usecolortheme{dove}
%\usecolortheme{fly}
%\usecolortheme{lily}
%\usecolortheme{orchid}
%\usecolortheme{rose}
%\usecolortheme{seagull}
%\usecolortheme{seahorse}
%\usecolortheme{whale}
%\usecolortheme{wolverine}

%\setbeamertemplate{footline} % To remove the footer line in all slides uncomment this line
%\setbeamertemplate{footline}[page number] % To replace the footer line in all slides with a simple slide count uncomment this line

%\setbeamertemplate{navigation symbols}{} % To remove the navigation symbols from the bottom of all slides uncomment this line
}

\usepackage{graphicx} % Allows including images
\usepackage{booktabs} % Allows the use of \toprule, \midrule and \bottomrule in tables
\usepackage{listings}

\lstdefinestyle{customjava}{
  breaklines=true,
  frame=L,
  xleftmargin=\parindent,
  language=Java,
  showstringspaces=false,
  basicstyle=\footnotesize\ttfamily,
  keywordstyle=\bfseries\color{green!40!black},
  commentstyle=\itshape\color{gray!40!black},
  identifierstyle=\color{blue},
  stringstyle=\color{orange},
}
%----------------------------------------------------------------------------------------
%	TITLE PAGE
%----------------------------------------------------------------------------------------

\title[Collections]{Collections} % The short title appears at the bottom of every slide, the full title is only on the title page

\author{Jonathan Windle} % Your name
\institute[UEA] % Your institution as it will appear on the bottom of every slide, may be shorthand to save space
{
University of East Anglia \\ % Your institution for the title page
\medskip
\textit{J.Windle@uea.ac.uk} % Your email address
}
\date{\today} % Date, can be changed to a custom date7

\begin{document}

\begin{frame}
\titlepage % Print the title page as the first slide
\end{frame}

\begin{frame}[allowframebreaks]
\frametitle{Overview} % Table of contents slide, comment this block out to remove it
\tableofcontents % Throughout your presentation, if you choose to use \section{} and \subsection{} commands, these will automatically be printed on this slide as an overview of your presentation
\end{frame}
%-----------------------------------------------------------------------
\section{Lists}
\subsection{ArrayList}
\begin{frame}
\frametitle{ArrayList}
\end{frame}

%--------------------------------------------------------------------
\defverbatim[colored]\sets{
\begin{lstlisting}[style = customjava,basicstyle=\scriptsize]
ArrayList<String> myList = new ArrayList<>();
myList.add("Bob");
myList.add("Bob"); // Won't add to the set
myList.add("Alice");
myList.add("Fred");
// Can use a List as a constructor argument
Set<String> mySet = new HashSet<>(myList); // Size is 3

if(mySet.add("Alice"))// Returns false as Alice already in set
	// Do things

\end{lstlisting}
}
\section{Sets}
\subsection{Using sets}
\begin{frame}
\frametitle{Using sets}
\begin{itemize}
\item Contains no duplicate elements in the collection.
\item Order of items also doesn't matter.
\sets
\end{itemize}
\end{frame}
%-----------------------------------------------------------------------
\subsection{Set Implementations}
\begin{frame}
\frametitle{Set Implementations}
\begin{itemize}
\item \texttt{\color{red}TreeSet}
\begin{itemize}
\item Stores the elements in a red-black tree, orders the elements based on their values.
\end{itemize}
\item \texttt{\color{green}HashSet}
\begin{itemize}
\item Stores elements in a \texttt{HashTable}
\end{itemize}
\item \texttt{\color{orange}LinkedHashSet}
\begin{itemize}
\item Implemented as a \texttt{HashTable} with a linked list running through it, orders elements in insertion order.
\end{itemize}
\end{itemize}
\end{frame}
%-------------------------------------------------------------------------
\subsection{TreeSet}
\begin{frame}
\frametitle{TreeSet}
\begin{itemize}
\item Stores elements in a \texttt{Tree} structure in order to maintain the elements in order.
\item \texttt{TreeSet} can be used with \texttt{Comparable} classes or by providing a \texttt{Comparator}.
\item These are useful when you need to extract elements from a set in a sorted manner.
\item \texttt{first()} - returns the smallest element.
\item \texttt{pollFirst()} which removes and returns the smallest element.
\item Search, insert and delete takes $O(logn)$ time.
\item Iterating in sorted order is $O(n)$ time iterating inorder traversal.
\end{itemize}
\end{frame}
%--------------------------------------------------------------------------
\subsection{HashSet}
\begin{frame}
\frametitle{HashSet}
\begin{itemize}
\item Stores the elements in a Hash Table.
\item Allows user to set initial {\color{red} capacity} and {\color{green} load factor}.
\item \texttt{HashSet} just contains a \texttt{HashMap} with keys defined by the \texttt{hashCode()}.
\item \texttt{HashSet} is generally the \texttt{Set} you would use unless you need to access the data in sorted order.
\item Insert, delete and contains are all $O(1)$ time.
\end{itemize}
\end{frame}
%---------------------------------------------------------------
\begin{frame}
\frametitle{HashSet - Behind the scenes}
\begin{enumerate}
\item Call the \texttt{hashCode()} function.
\item Apply the \texttt{hash()} function to find the index.
\item If the position index is empty, add to the head of the list, otherwise find the end of the list and add there.
\end{enumerate}
\begin{itemize}
\item After the \texttt{hash} is complete it is then bitwise \& with $arrLength - 1$ where $arrLength$ is a power of two. 
\item $71638\%8$ is equivalent to $71638\&7$ and compensates for negative hash codes.
\end{itemize}
\end{frame}
%-----------------------------------------------------------------
\subsubsection{HashSet notes}
\begin{frame}
\frametitle{HashSet notes}
\begin{itemize}
\item Two parameters that affect its performance:
\begin{itemize}
\item {\color{red} initial capacity} (Default 16, increases to nearest power of 2).
\item {\color{green} load factor} (Default to 0.75). This determines when to resize the array, e.g. with capacity 16, when the 12th element is added, the array is resized, when this happens:
\begin{itemize}
\item All indeces are recalculated
\item Chains are removed/reduced.
\item This is expensive.
\end{itemize}
\end{itemize}
\item The performance of the \texttt{hash} function has a large effect on the overall performance.
\item As chains get long, performance decreases to $O(m)$ 
\item Extra efficiency is obtained at the cost of memory.
\item Useful if you want to query a lot, using \texttt{contains}. 
\end{itemize}
\end{frame}
%---------------------------------------------------------------------
\subsection{LinkedHashSet}
\begin{frame}
\frametitle{LinkedHashSet}
\begin{itemize}
\item Is the same as HashSet, however includes a {\color{red} doubly linked list}. This means it can be traversed in insertion order too.
\end{itemize}
\end{frame}
%-----------------------------------------------------------------------
\section{Map}
\begin{frame}
\frametitle{Map}
\begin{itemize}
\item Allows you to associate a key with one or more values and then quickly retrieve those values using the key.
\item A map cannot contain duplicate keys: Each key can map to at most one value. \textbf{It can contain duplicate values}.
\item A set is simply a map with the key equal to the value.
\end{itemize}
\textbf{
Set : \{Fred, Alice, Bob\}\\
Map:  \{(1,Fred), (2,Alice), (3,Bob)\}\\
Set as a Map: \{(Fred, Fred), (Alice, Alice), (Bob, Bob)\}\\}
\begin{itemize}
\item Maps are a collection of \texttt{Entry} objects, these store both the key and the value for a given map entry.
\end{itemize}
\end{frame}
%-------------------------------------------------------------------
\defverbatim[colored]\hash{
\begin{lstlisting}[style = customjava,basicstyle=\scriptsize]
// String key and Student value
HashMap<String, Student> hm = new HashMap<>();
hm.put("BobKey", new Student(33, "Bob"));
\end{lstlisting}
}
\subsubsection{HashMap}
\begin{frame}
\frametitle{HashMap}
\hash
\begin{enumerate}
\item A new \texttt{Entry} object is created containing the key and the value.
\item The \texttt{hashCode} function is called on the key to find the location in the hash table.
\end{enumerate}
\texttt{HashSet} is just a \texttt{HashMap} where the value is used as the key.
\end{frame}
%------------------------------------------------------------
\subsubsection{TreeMap}
\begin{frame}
\frametitle{TreeMap}
\begin{itemize}
\item Behaves just as the \texttt{TreeSet} class, except the key is used to determine the location of the value.
\end{itemize}
\begin{enumerate}
\item The key must be comparable.
\item the \texttt{compareTo} method is used to insert the entry to the correct position.
\item \texttt{TreeSet} simply contains a \texttt{TreeMap} with the values set to the key. 
\end{enumerate}
\end{frame}
%----------------------------------------------------------------
\section{Queues and Deques}
\subsection{Priority Queues}
\begin{frame}
\frametitle{Queues and Deques}
\begin{itemize}
\item Java uses a {\color{green}linked list} for queues and deques.
\item Priority queues store elements in a {\color{red}heap} data structure (a complete binary tree).
\item Duplicates are allowed.
\item $O(logn)$ time for insertion methods (\texttt{offer,poll, remove and add})
\item $O(n)$ time for \texttt{remove(Object)} and \texttt{contains}
\item $(1)$ time for retrieval methods, \texttt{peek, element and size}.
\item Iterator not guaranteed to traverse in any order.
\end{itemize}
\end{frame}
%-------------------------------------------------------------------
\section{Collections class}
\begin{frame}
\begin{itemize}
\item Useful methods:
\begin{itemize}
\item \texttt{frequency(Collection<?>c, Object o)} - Returns occurences of o in c.
\item \texttt{rotate(Collection<?> list, int distance)} - Shifts all the elements right in a logical circular way.
\item \texttt{shuffle(List<?>, list)} - Performs n normal swaps.
\end{itemize}
\item Sorting:
\begin{itemize}
\item \texttt{sort(List<T> list)} - A modified merge sort.
\item Mergesort that does not perform the merge operation if not required.
\item It dumps the specified list into an array,sorts the array, and iterates over the list resetting each element from the corresponding position in the array.
\item This avoids the $n^2log(n)$ performance that would result from attempting to sort a linked list in place.
\end{itemize}
\end{itemize}
\end{frame}
%----------------------------------------------------------------------------------
\begin{frame}
\Huge{\centerline{The End}}
\end{frame}

%----------------------------------------------------------------------------------------

\end{document}