%%%%%%%%%%%%%%%%%%%%%%%%%%%%%%%%%%%%%%%%%
% Beamer Presentation
% LaTeX Template
% Version 1.0 (10/11/12)
%
% This template has been downloaded from:
% http://www.LaTeXTemplates.com
%
% License:
% CC BY-NC-SA 3.0 (http://creativecommons.org/licenses/by-nc-sa/3.0/)
%
%%%%%%%%%%%%%%%%%%%%%%%%%%%%%%%%%%%%%%%%%

%----------------------------------------------------------------------------------------
%	PACKAGES AND THEMES
%----------------------------------------------------------------------------------------

\documentclass{beamer}

\mode<presentation> {

% The Beamer class comes with a number of default slide themes
% which change the colors and layouts of slides. Below this is a list
% of all the themes, uncomment each in turn to see what they look like.

%\usetheme{default}
%\usetheme{AnnArbor}
%\usetheme{Antibes}
%\usetheme{Bergen}
%\usetheme{Berkeley}
%\usetheme{Berlin}
%\usetheme{Boadilla}
%\usetheme{CambridgeUS}
%\usetheme{Copenhagen}
%\usetheme{Darmstadt}
%\usetheme{Dresden}
%\usetheme{Frankfurt}
%\usetheme{Goettingen}
%\usetheme{Hannover}
%\usetheme{Ilmenau}
%\usetheme{JuanLesPins}
%\usetheme{Luebeck}
\usetheme{Madrid}
%\usetheme{Malmoe}
%\usetheme{Marburg}
%\usetheme{Montpellier}
%\usetheme{PaloAlto}
%\usetheme{Pittsburgh}
%\usetheme{Rochester}
%\usetheme{Singapore}
%\usetheme{Szeged}
%\usetheme{Warsaw}

% As well as themes, the Beamer class has a number of color themes
% for any slide theme. Uncomment each of these in turn to see how it
% changes the colors of your current slide theme.

%\usecolortheme{albatross}
%\usecolortheme{beaver}
%\usecolortheme{beetle}
%\usecolortheme{crane}
%\usecolortheme{dolphin}
%\usecolortheme{dove}
%\usecolortheme{fly}
%\usecolortheme{lily}
%\usecolortheme{orchid}
%\usecolortheme{rose}
%\usecolortheme{seagull}
%\usecolortheme{seahorse}
%\usecolortheme{whale}
%\usecolortheme{wolverine}

%\setbeamertemplate{footline} % To remove the footer line in all slides uncomment this line
%\setbeamertemplate{footline}[page number] % To replace the footer line in all slides with a simple slide count uncomment this line

%\setbeamertemplate{navigation symbols}{} % To remove the navigation symbols from the bottom of all slides uncomment this line
}

\usepackage{graphicx} % Allows including images
\usepackage{booktabs} % Allows the use of \toprule, \midrule and \bottomrule in tables
\usepackage{listings}
\usepackage{amsmath}
\usepackage{algpseudocode,algorithm,algorithmicx}

\lstdefinestyle{customjava}{
  breaklines=true,
  frame=L,
  xleftmargin=\parindent,
  language=Java,
  showstringspaces=false,
  basicstyle=\footnotesize\ttfamily,
  keywordstyle=\bfseries\color{green!40!black},
  commentstyle=\itshape\color{gray!40!black},
  identifierstyle=\color{blue},
  stringstyle=\color{orange},
}


%----------------------------------------------------------------------------------------
%	TITLE PAGE
%----------------------------------------------------------------------------------------

\title[Iterators]{Iterators} % The short title appears at the bottom of every slide, the full title is only on the title page

\author{Jonathan Windle} % Your name
\institute[UEA] % Your institution as it will appear on the bottom of every slide, may be shorthand to save space
{
University of East Anglia \\ % Your institution for the title page
\medskip
\textit{J.Windle@uea.ac.uk} % Your email address
}
\date{\today} % Date, can be changed to a custom date

\begin{document}

\begin{frame}
\titlepage % Print the title page as the first slide
\end{frame}

\begin{frame}[allowframebreaks]
\frametitle{Overview} % Table of contents slide, comment this block out to remove it
\tableofcontents % Throughout your presentation, if you choose to use \section{} and \subsection{} commands, these will automatically be printed on this slide as an overview of your presentation
\end{frame}

%-----------------------------------------------------------------
\section{Intro}
\begin{frame}
\frametitle{Intro}
\begin{itemize}
\item Iterate over a collection of objects.
\item Uses the \texttt{Iterator} interface.

\end{itemize}
\end{frame}
%----------------------------------------------------------------
\defverbatim[colored]\create{
\begin{lstlisting}[style = customjava]
Iterator it = list.iterator(); // Raw type iterator
Iterator<Integer> it2 = list.iterator();
Iterator<Double> it3 = arr.iterator();
Iterator<Card> it4 = deck.iterator();
// Can hold arr or list iterator
Iterator<? extends Number> it5 = arr.iterator(); 
\end{lstlisting}
}
\section{Using iterators}
\subsection{Creating instance}
\begin{frame}
\frametitle{Using iterators - creating instance}
\begin{itemize}
\item Create an instance like this:
\create
\end{itemize}
\end{frame}
%-----------------------------------------------------------------
\defverbatim[colored]\iter{
\begin{lstlisting}[style = customjava]
Iterator it = list.iterator(); // Raw type iterator
Iterator<Integer> castEx = list.iterator();
Integer anInt;
// Returns true if any elements are left to iterate over
while (it.hasNext()) {
// Gets the next element and moves the iterator forward.
	anInt = (Integer) it.next(); // Remove cast by using generics
	System.out.println("Element = " + anInt);
	it.remove();  // Removes last element returned by iterator.
}
\end{lstlisting}
}
\subsection{Iterating}
\begin{frame}
\frametitle{Iterating}
\iter
\end{frame}
%-----------------------------------------------------------------
\defverbatim[colored]\foreach{
\begin{lstlisting}[style = customjava, basicstyle=\tiny]
Iterator<Integer> it = list.iterator();
while (it.hasNext()) {
	System.out.println(it.next());
}
// Is equal to:
for(Integer i : list) // list must be iterable to use in foreach
	System.out.println(i);
\end{lstlisting}
}
\begin{frame}
\frametitle{Iterator Errata}
\begin{itemize}
\item \texttt{List} has a special Iterator. The interface \texttt{ListIterator} extends the\texttt{Iterator} interface to include extra methods to move backwards in the list:
\begin{itemize}
\item \texttt{previous()}
\item \texttt{hasPrevious()}
\item \texttt{add(E o)}
\item \texttt{nextIndex()}
\item \texttt{previousIndex()}
\item \texttt{set(E o)}
\end{itemize}
\item For-each control structure uses \texttt{Iterator} to do so.
\foreach
\item Cannot us for-each to structually modify the collection.
\end{itemize}
\end{frame}
%-----------------------------------------------------------------
\defverbatim[colored]\cust{
\begin{lstlisting}[style = customjava, basicstyle=\tiny]
public class SortedList<T> implements Iterable<T>

	private class SortedIterator implements Iterator<T> {
		int pos = 0;
		@override
		public boolean hasNext() {
			return pos < size; // Size of array
		}
		@override
		public T next() {
			return data[pos++];
		}
		@override
		public void remove() {
			// Optional
		}

	}
}
\end{lstlisting}
}
\section{Creating own Iterators}
\begin{frame}
\frametitle{Creating own Iterators}
\begin{itemize}
\item They are defined as {\color{red} private inner classes}
\item Class should implement the \texttt{Iterable} interface, which requires a \texttt{iterator()} method.
\item When implemented the class can be used in a foreach loop.
\cust
\end{itemize}
\end{frame}
%-----------------------------------------------------------------
\begin{frame} 
\Huge{\centerline{The End}}
\end{frame}

\end{document}