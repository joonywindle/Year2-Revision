%%%%%%%%%%%%%%%%%%%%%%%%%%%%%%%%%%%%%%%%%
% Beamer Presentation
% LaTeX Template
% Version 1.0 (10/11/12)
%
% This template has been downloaded from:
% http://www.LaTeXTemplates.com
%
% License:
% CC BY-NC-SA 3.0 (http://creativecommons.org/licenses/by-nc-sa/3.0/)
%
%%%%%%%%%%%%%%%%%%%%%%%%%%%%%%%%%%%%%%%%%

%----------------------------------------------------------------------------------------
%	PACKAGES AND THEMES
%----------------------------------------------------------------------------------------

\documentclass{beamer}

\mode<presentation> {

% The Beamer class comes with a number of default slide themes
% which change the colors and layouts of slides. Below this is a list
% of all the themes, uncomment each in turn to see what they look like.

%\usetheme{default}
%\usetheme{AnnArbor}
%\usetheme{Antibes}
%\usetheme{Bergen}
%\usetheme{Berkeley}
%\usetheme{Berlin}
%\usetheme{Boadilla}
%\usetheme{CambridgeUS}
%\usetheme{Copenhagen}
%\usetheme{Darmstadt}
%\usetheme{Dresden}
%\usetheme{Frankfurt}
%\usetheme{Goettingen}
%\usetheme{Hannover}
%\usetheme{Ilmenau}
%\usetheme{JuanLesPins}
%\usetheme{Luebeck}
\usetheme{Madrid}
%\usetheme{Malmoe}
%\usetheme{Marburg}
%\usetheme{Montpellier}
%\usetheme{PaloAlto}
%\usetheme{Pittsburgh}
%\usetheme{Rochester}
%\usetheme{Singapore}
%\usetheme{Szeged}
%\usetheme{Warsaw}

% As well as themes, the Beamer class has a number of color themes
% for any slide theme. Uncomment each of these in turn to see how it
% changes the colors of your current slide theme.

%\usecolortheme{albatross}
%\usecolortheme{beaver}
%\usecolortheme{beetle}
%\usecolortheme{crane}
%\usecolortheme{dolphin}
%\usecolortheme{dove}
%\usecolortheme{fly}
%\usecolortheme{lily}
%\usecolortheme{orchid}
%\usecolortheme{rose}
%\usecolortheme{seagull}
%\usecolortheme{seahorse}
%\usecolortheme{whale}
%\usecolortheme{wolverine}

%\setbeamertemplate{footline} % To remove the footer line in all slides uncomment this line
%\setbeamertemplate{footline}[page number] % To replace the footer line in all slides with a simple slide count uncomment this line

%\setbeamertemplate{navigation symbols}{} % To remove the navigation symbols from the bottom of all slides uncomment this line
}

\usepackage{graphicx} % Allows including images
\usepackage{booktabs} % Allows the use of \toprule, \midrule and \bottomrule in tables
\usepackage{listings}
\usepackage{amsmath}
\usepackage{algpseudocode,algorithm,algorithmicx}

\lstdefinestyle{customjava}{
  breaklines=true,
  frame=L,
  xleftmargin=\parindent,
  language=Java,
  showstringspaces=false,
  basicstyle=\footnotesize\ttfamily,
  keywordstyle=\bfseries\color{green!40!black},
  commentstyle=\itshape\color{gray!40!black},
  identifierstyle=\color{blue},
  stringstyle=\color{orange},
}

%----------------------------------------------------------------------------------------
%	TITLE PAGE
%----------------------------------------------------------------------------------------

\title[Serialisation]{Serialization} % The short title appears at the bottom of every slide, the full title is only on the title page

\author{Jonathan Windle} % Your name
\institute[UEA] % Your institution as it will appear on the bottom of every slide, may be shorthand to save space
{
University of East Anglia \\ % Your institution for the title page
\medskip
\textit{J.Windle@uea.ac.uk} % Your email address
}
\date{\today} % Date, can be changed to a custom date

\begin{document}

\begin{frame}
\titlepage % Print the title page as the first slide
\end{frame}

\begin{frame}[allowframebreaks]
\frametitle{Overview} % Table of contents slide, comment this block out to remove it
\tableofcontents % Throughout your presentation, if you choose to use \section{} and \subsection{} commands, these will automatically be printed on this slide as an overview of your presentation
\end{frame}

%------------------------------------------------------------------
\section{Intro}
\begin{frame}
\frametitle{Intro}
\begin{itemize}
\item Provides the ability to save the {\color{red} state} of an {\color{green} object} beyond the life of the program and the virtual machine.
\item {\color{green} Objects} are flattened into {\color{purple} byte code} so they can be easily loaded later.
\item Class to be serialized must implement the \texttt{Serializable} interface.
\end{itemize}
\end{frame}
%-----------------------------------------------------------------
\section{Default Serialization}
\begin{frame}
\frametitle{Default Serialization}
\begin{itemize}
\item Can persist objects to a database using JDBC or across a network.
\item \textbf{Cannot} persist {\color{red} static} fields.
\item \texttt{Object} doesn't implement \texttt{Serializable}.
\item If an \texttt{Object} is \texttt{Serializable}, by default all fields are saved.
\item When read in, the Constructor is not called.
\item If class structure has changed, for example, fields or methods change, an \textbf{\texttt{InvalidClassException}} is thrown.
\item Repeated writes do not overwrite previous writes, you must close and reopen to do so.
\item By default if a class is \texttt{\color{green} Iterable} it uses that to store the whole class.
\end{itemize}
\end{frame}

%-----------------------------------------------------------------
\subsection{Version Control}
\begin{frame}
\frametitle{Version Control}
\begin{itemize}
\item All serialized classes contain a \texttt{serialVersionUID}.
\item \texttt{serialVersionUID} is used by \texttt{readObject()} to check it's ok to load. 
\item If \texttt{serialVersionUID} is not provided, it is defaulted to a sum of hashCodes.
\item If \texttt{serialVersionUID} is set, you can still load the .ser file even if fields have been added. If a conflict occurs and are incompatible, an exception is thrown.
\item If \texttt{serialVersionUID} doesn't match that of the .ser file, then it won't be loaded.
\item \texttt{{\color{red} transient}} variables are not serialized with the object. It remains null until set in the loaded system.
\end{itemize}
\end{frame}
%----------------------------------------------------------------
\section{Customised Serialization}
\begin{frame}
\frametitle{Customised Serialization}
\begin{itemize}
\item Easy to control how an \texttt{Object} is \texttt{serialized} by implementing \texttt{\color{red}writeObject()} and \texttt{\color{green}readObject()}.
\item Still called as they are in the default manor, just the JVM checks if they are implemented or not and uses them if they are.
\item They must be \texttt{\color{blue}private} so they cannot be overridden.
\end{itemize}
\end{frame}
%-----------------------------------------------------------------
\defverbatim[colored]\example{
\begin{lstlisting}[style = customjava,basicstyle=\scriptsize]
public class HashMap<K,V> extends AbstractMap<K,V>
implements Map<K,V>, Cloneable, Serializable
{
transient Entry[] table; // Don't want to store the empty entries
private void writeObject(java.io.ObjectOutputStream s)
         throws IOException
     {
     // Write all the non-transient data using normal mehtod
	s.defaultWriteObject();
	s.writeInt(table.length); // Write number of buckets
	s.writeInt(size); // Write out number of mappings


	Iterator<Map.Entry<K,V>> 
	// Returns set view of the map
	i = (size > 0) ? entrySet().iterator() : null; 
	// Iterate over elements.
   	if (i != null) {
		while (i.hasNext()) {
      		Map.Entry<K,V> e = i.next();
      		// Write objects
			s.writeObject(e.getKey()); 
			s.writeObject(e.getValue());
     	}
  	}
  	}
  	}
\end{lstlisting}
}
\subsection{Example}
\begin{frame}
\frametitle{Example}
\example
\end{frame}
%-----------------------------------------------------------------
\subsection{Caching}
\begin{frame}
\frametitle{Caching}
\begin{itemize}
\item Cannot overwrite an object once saved without closing and opening again.
\item Solution:
\begin{enumerate}
\item Always close and repoen.
\item Flush the cache by calling \texttt{out.reset()}.
\end{enumerate}
\end{itemize}
\end{frame}
%-----------------------------------------------------------------


\begin{frame} 
\Huge{\centerline{The End}}
\end{frame}

\end{document}