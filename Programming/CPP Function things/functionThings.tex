%%%%%%%%%%%%%%%%%%%%%%%%%%%%%%%%%%%%%%%%%
% Beamer Presentation
% LaTeX Template
% Version 1.0 (10/11/12)
%
% This template has been downloaded from:
% http://www.LaTeXTemplates.com
%
% License:
% CC BY-NC-SA 3.0 (http://creativecommons.org/licenses/by-nc-sa/3.0/)
%
%%%%%%%%%%%%%%%%%%%%%%%%%%%%%%%%%%%%%%%%%

%----------------------------------------------------------------------------------------
%	PACKAGES AND THEMES
%----------------------------------------------------------------------------------------

\documentclass{beamer}

\mode<presentation> {

% The Beamer class comes with a number of default slide themes
% which change the colors and layouts of slides. Below this is a list
% of all the themes, uncomment each in turn to see what they look like.

%\usetheme{default}
%\usetheme{AnnArbor}
%\usetheme{Antibes}
%\usetheme{Bergen}
%\usetheme{Berkeley}
%\usetheme{Berlin}
%\usetheme{Boadilla}
%\usetheme{CambridgeUS}
%\usetheme{Copenhagen}
%\usetheme{Darmstadt}
%\usetheme{Dresden}
%\usetheme{Frankfurt}
%\usetheme{Goettingen}
%\usetheme{Hannover}
%\usetheme{Ilmenau}
%\usetheme{JuanLesPins}
%\usetheme{Luebeck}
\usetheme{Madrid}
%\usetheme{Malmoe}
%\usetheme{Marburg}
%\usetheme{Montpellier}
%\usetheme{PaloAlto}
%\usetheme{Pittsburgh}
%\usetheme{Rochester}
%\usetheme{Singapore}
%\usetheme{Szeged}
%\usetheme{Warsaw}

% As well as themes, the Beamer class has a number of color themes
% for any slide theme. Uncomment each of these in turn to see how it
% changes the colors of your current slide theme.

%\usecolortheme{albatross}
%\usecolortheme{beaver}
%\usecolortheme{beetle}
%\usecolortheme{crane}
%\usecolortheme{dolphin}
%\usecolortheme{dove}
%\usecolortheme{fly}
%\usecolortheme{lily}
%\usecolortheme{orchid}
%\usecolortheme{rose}
%\usecolortheme{seagull}
%\usecolortheme{seahorse}
%\usecolortheme{whale}
%\usecolortheme{wolverine}

%\setbeamertemplate{footline} % To remove the footer line in all slides uncomment this line
%\setbeamertemplate{footline}[page number] % To replace the footer line in all slides with a simple slide count uncomment this line

%\setbeamertemplate{navigation symbols}{} % To remove the navigation symbols from the bottom of all slides uncomment this line
}

\usepackage{graphicx} % Allows including images
\usepackage{booktabs} % Allows the use of \toprule, \midrule and \bottomrule in tables
\usepackage{listings}
\usepackage{amsmath}
\usepackage{algpseudocode,algorithm,algorithmicx}

\lstdefinestyle{customjava}{
  breaklines=true,
  frame=L,
  xleftmargin=\parindent,
  language=Java,
  showstringspaces=false,
  basicstyle=\footnotesize\ttfamily,
  keywordstyle=\bfseries\color{green!40!black},
  commentstyle=\itshape\color{gray!40!black},
  identifierstyle=\color{blue},
  stringstyle=\color{orange},
}

\lstdefinestyle{customcpp}{
  breaklines=true,
  frame=L,
  xleftmargin=\parindent,
  language=C++,
  showstringspaces=false,
  basicstyle=\footnotesize\ttfamily,
  keywordstyle=\bfseries\color{green!40!black},
  commentstyle=\itshape\color{gray!40!black},
  identifierstyle=\color{blue},
  stringstyle=\color{orange},
}
%----------------------------------------------------------------------------------------
%	TITLE PAGE
%----------------------------------------------------------------------------------------

\title[Function things]{Function things} % The short title appears at the bottom of every slide, the full title is only on the title page

\author{Jonathan Windle} % Your name
\institute[UEA] % Your institution as it will appear on the bottom of every slide, may be shorthand to save space
{
University of East Anglia \\ % Your institution for the title page
\medskip
\textit{J.Windle@uea.ac.uk} % Your email address
}
\date{\today} % Date, can be changed to a custom date

\begin{document}

\begin{frame}
\titlepage % Print the title page as the first slide
\end{frame}

\begin{frame}[allowframebreaks]
\frametitle{Overview} % Table of contents slide, comment this block out to remove it
\tableofcontents % Throughout your presentation, if you choose to use \section{} and \subsection{} commands, these will automatically be printed on this slide as an overview of your presentation
\end{frame}

%-------------------------------------------------------------------
\defverbatim[colored]\fun{
\begin{lstlisting}[style = customcpp]
template<typename T, int N, typename F>
void display(T (&array)[N], F& hasProperty) {
	for (int i = 0; i < N; i++) {
		if (hasProperty(array[i]))
			cout << array[i] << " ";
	}
}
\end{lstlisting}
}
\section{Functors}
\begin{frame}
\frametitle{Functors}
\begin{itemize}
\item Is an object that acts like a function.
\item Provides plug-in that adapts the behaviour of other methods/objects.
\item E.g Print all elements in an array that have some property:
\begin{itemize}
\item Problem: Don't know what kind of properties.
\item Solution: Supply an object as a parameter that has:
\begin{itemize}
\item A method to determine whether an element has a property.
\item Instance variables that describe the particular property.
\end{itemize}
\end{itemize}
\end{itemize}
\fun
\end{frame}
%-------------------------------------------------------------------
\defverbatim[colored]\ismul{
\begin{lstlisting}[style = customcpp]
template <typename T>
class IsMultipleOf {
private:
	T n;
public:
	IsMultipleOf(T n){ this->n = n; }
// Use object like a function
	bool operator()(T& v){
		return v%n == 0;
	}
}
int main(int argc, char* argv) {
	int array[] = {+2,-6,+3,-5,+7,-9,+3,-2,+4,-5,+8,-8,+9};
	IsMultipleOf<int> isMultiple(3);
// Prints all multiples of 3
	display(array, isMultiple);
	return 0;
}
\end{lstlisting}
}
\subsection{Example}
\begin{frame}
\frametitle{Example}
Determine if a number is a fixed multiple of some integer:
\ismul
\end{frame}

%-----------------------------------------------------------------
\defverbatim[colored]\lam{
\begin{lstlisting}[style = customcpp]
[N](int n) const char* {return (n%N==0)?"even":"odd"}
\end{lstlisting}
}
\section{Lambdas}
\begin{frame}
\frametitle{Lambdas}
\begin{itemize}
\item Basically an anonymous function
\item Alternative to functors.
\lam
\item Have four parts:
\begin{itemize}
\item Capture list: \texttt{[N]}
\item Parameter list: \texttt{(int n)}
\item Optional return type, compiler will deduce one if not specified: \texttt{const char*}
\item The body: \texttt{\{return (n\%N==0)?"even":"odd"\}}
\end{itemize}
\item Can capture local and global variables of the scope in which they are declared.
\end{itemize}
\end{frame}
%------------------------------------------------------------------
\defverbatim[colored]\lamex{
\begin{lstlisting}[style = customcpp]

int main(int argc, char* argv) {
	int array[] = {+2,-6,+3,-5,+7,-9,+3,-2,+4,-5,+8,-8,+9};

// Prints all even numbers
	display(array, [](int n){return !(n%2);});
	return 0;
}
\end{lstlisting}
}
\subsection{Example}
\begin{frame}
\frametitle{Example}
\begin{itemize}
\item Passed to templates the same way as Functors:
\lamex
\end{itemize}
\end{frame}
%-----------------------------------------------------------------
\defverbatim[colored]\val{
\begin{lstlisting}[style = customcpp]

int main(int argc, char* argv) {
	int array[] = {+2,-6,+3,-5,+7,-9,+3,-2,+4,-5,+8,-8,+9};
	int N = 3;
	
// Prints all multiples of 3, captured by value
	display(array, [N](int n){return !(n%N);});
	return 0;
}
\end{lstlisting}
}
\subsection{Capture by value}
\begin{frame}
\frametitle{Capture by value}
\begin{itemize}
\item Simplest/Safest way is to capture by value, therefore a copy is passed to the lambda.
\val
\end{itemize}
\end{frame}
%-----------------------------------------------------------------
\defverbatim[colored]\ree{
\begin{lstlisting}[style = customcpp]

int main(int argc, char* argv) {
	int array[] = {+2,-6,+3,-5,+7,-9,+3,-2,+4,-5,+8,-8,+9};
	int N = 1;
	
// Prints all multiples of N where n doubles each iteration?
	display(array, [&N](int n){return !(n%N*2);});
// N is 2,4,8,16.... etc..
	return 0;
}
\end{lstlisting}
}
\subsection{Capture by reference}
\begin{frame}
\frametitle{Capture by reference}
\begin{itemize}
\item Lambda can in turn, alter the value taken in by reference using \&.
\ree
\item Default is capture by value, unless specified by \&. Can use:
\begin{itemize}
\item {[\&, \textit{list...}]}, to capture all by reference unless otherwise specified.
\end{itemize}
\end{itemize}
\end{frame}
%-----------------------------------------------------------------
\defverbatim[colored]\aut{
\begin{lstlisting}[style = customcpp]
auto multipleOfN = [N](int n){if (n%N == 0) cout << n;};
\end{lstlisting}
}
\defverbatim[colored]\std{
\begin{lstlisting}[style = customcpp]
std::function<boolean(int)> multipleOfN = [N](int n) boolean {if (n%N == 0) cout << n;};
\end{lstlisting}
}
\subsection{Storing lambdas}
\begin{frame}
\frametitle{Storing lambdas}
\begin{itemize}
\item Can use \texttt{\color{blue} auto}, leaves to the compiler to deduce type.
\aut
\item Alternatively can use \texttt{std::function<R(AL)>}
\begin{itemize}
\item R is the return type.
\item AL is comma seperated list of arguments
\std
\end{itemize}
\end{itemize}
\end{frame}
%------------------------------------------------------------------

\begin{frame} 
\Huge{\centerline{The End}}
\end{frame}

\end{document}