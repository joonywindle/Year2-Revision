%%%%%%%%%%%%%%%%%%%%%%%%%%%%%%%%%%%%%%%%%
% Beamer Presentation
% LaTeX Template
% Version 1.0 (10/11/12)
%
% This template has been downloaded from:
% http://www.LaTeXTemplates.com
%
% License:
% CC BY-NC-SA 3.0 (http://creativecommons.org/licenses/by-nc-sa/3.0/)
%
%%%%%%%%%%%%%%%%%%%%%%%%%%%%%%%%%%%%%%%%%

%----------------------------------------------------------------------------------------
%	PACKAGES AND THEMES
%----------------------------------------------------------------------------------------

\documentclass{beamer}

\mode<presentation> {

% The Beamer class comes with a number of default slide themes
% which change the colors and layouts of slides. Below this is a list
% of all the themes, uncomment each in turn to see what they look like.

%\usetheme{default}
%\usetheme{AnnArbor}
%\usetheme{Antibes}
%\usetheme{Bergen}
%\usetheme{Berkeley}
%\usetheme{Berlin}
%\usetheme{Boadilla}
%\usetheme{CambridgeUS}
%\usetheme{Copenhagen}
%\usetheme{Darmstadt}
%\usetheme{Dresden}
%\usetheme{Frankfurt}
%\usetheme{Goettingen}
%\usetheme{Hannover}
%\usetheme{Ilmenau}
%\usetheme{JuanLesPins}
%\usetheme{Luebeck}
\usetheme{Madrid}
%\usetheme{Malmoe}
%\usetheme{Marburg}
%\usetheme{Montpellier}
%\usetheme{PaloAlto}
%\usetheme{Pittsburgh}
%\usetheme{Rochester}
%\usetheme{Singapore}
%\usetheme{Szeged}
%\usetheme{Warsaw}

% As well as themes, the Beamer class has a number of color themes
% for any slide theme. Uncomment each of these in turn to see how it
% changes the colors of your current slide theme.

%\usecolortheme{albatross}
%\usecolortheme{beaver}
%\usecolortheme{beetle}
%\usecolortheme{crane}
%\usecolortheme{dolphin}
%\usecolortheme{dove}
%\usecolortheme{fly}
%\usecolortheme{lily}
%\usecolortheme{orchid}
%\usecolortheme{rose}
%\usecolortheme{seagull}
%\usecolortheme{seahorse}
%\usecolortheme{whale}
%\usecolortheme{wolverine}

%\setbeamertemplate{footline} % To remove the footer line in all slides uncomment this line
%\setbeamertemplate{footline}[page number] % To replace the footer line in all slides with a simple slide count uncomment this line

%\setbeamertemplate{navigation symbols}{} % To remove the navigation symbols from the bottom of all slides uncomment this line
}

\usepackage{graphicx} % Allows including images
\usepackage{booktabs} % Allows the use of \toprule, \midrule and \bottomrule in tables
\usepackage{listings}
\usepackage{amsmath}
\usepackage{algpseudocode,algorithm,algorithmicx}


%----------------------------------------------------------------------------------------
%	TITLE PAGE
%----------------------------------------------------------------------------------------

\title[Lists Stacks \& Queues]{Lists Stacks \& Queues} % The short title appears at the bottom of every slide, the full title is only on the title page

\author{Jonathan Windle} % Your name
\institute[UEA] % Your institution as it will appear on the bottom of every slide, may be shorthand to save space
{
University of East Anglia \\ % Your institution for the title page
\medskip
\textit{J.Windle@uea.ac.uk} % Your email address
}
\date{\today} % Date, can be changed to a custom date

\begin{document}

\begin{frame}
\titlepage % Print the title page as the first slide
\end{frame}

\begin{frame}[allowframebreaks]
\frametitle{Overview} % Table of contents slide, comment this block out to remove it
\tableofcontents % Throughout your presentation, if you choose to use \section{} and \subsection{} commands, these will automatically be printed on this slide as an overview of your presentation
\end{frame}

%------------------------------------------------------------------
\section{Lists}
\subsection{Comparisons}
\begin{frame}
\frametitle{Comparisons}
\begin{itemize}
\item Is a {\color{purple} linear data structure}.
\item A List is a collection where the elements are {\color{red} ordered} and therefore each element has an index which is the position in the list. Allows duplicates.
\item A Set is  an {\color{green} unordered} collection in which no two elements are identical.
\item A Bag is an {\color{green} unordered} collection in which can have duplicates.
\end{itemize}
\begin{tabular}{|c|c|c|}
\hline
 & Linked List & Array based \\
\hline
 Access & $\Theta(n)$ & $\Theta(1)$ \\
 \hline
 Insertion & $O(n)$ & $O(1)$\\
 \hline
 Deletion & $O(n)$ & $O(n)$\\
 \hline
\end{tabular}
\end{frame}
%------------------------------------------------------------------
\subsection{Amortized Analysis}
\begin{frame}
\frametitle{Amortized Analysis}
\begin{itemize}
\item In {\color{green} Amortized analysis}, the time taken to execute a sequence is averaged over all the operations executed.
\item Even though one of the operations in the sequence might take a long time, the average time taken over all operations is small.
\item This is not the same as the average case.
\item This guarantees the average performance of each operation in the worst-case.
\end{itemize}
\end{frame}

%-----------------------------------------------------------------
\section{Stacks}
\subsection{Intro}
\begin{frame}
\frametitle{Intro}
\begin{itemize}
\item It's a list structure where all operations occur on one end of the list, known as the {\color{green} top of the stack}.
\item To add an element is called a {\color{red} push} operation.
\item To remove an element is called a {\color{purple} pop} operation.
\item To get the element at the {\color{green} top of the stack} is called a {\color{orange} top} operation. 
\end{itemize}
\end{frame}

%----------------------------------------------------------------
\subsection{Array implementation}
\begin{frame}
\frametitle{Array implementation}
\begin{itemize}
\item Requires a means of handling {\color{red} array overflow}, i.e. double size of array when full.
\item \texttt{push()} has an {\color{green} amortized} complexity of $O(1)$ in the worst case.
\item \texttt{top()} does not alter the stack at all and simply gives the top element, this is $O(1)$ in the worst case.
\item \texttt{pop()} only alters the last element, nothing is shifted and therefore has complexity $O(1)$ in the worst case.
\end{itemize}
\end{frame}
%------------------------------------------------------------------
\subsection{Linked-list implementation}
\begin{frame}
\frametitle{Linked-list implementation}
\begin{itemize}
\item \texttt{push()} has a complexity of $O(1)$ in the worst case, this is {\color{red} NOT} {\color{green} amortized} due to the lack of array overflow requirement.
\item \texttt{top()} has $O(1)$ complexity in the worst case.
\item \texttt{pop()} has $O(1)$ complexity in the worst case.
\end{itemize}
\end{frame}
%-----------------------------------------------------------------
\subsection{Parenthesis checking}
\begin{frame}
\frametitle{Parenthesis checking}
\begin{itemize}
\item Stacks can be used to determine is parenthasis match correctly or not.
e.g. $[
a
∗
(
b
+
c
)
−{
d
−
a
}
/c
+
e
]
/b$
\item Use a stack to push the left side of the parenthesis and when the right side has been found, pop the parenthasis.
\item When an item is popped, it is compared to the found parenthesis and if they are of the same type, then it's matching.

\end{itemize}
\end{frame}

%---------------------------------------------------------------
\section{Queues}
\subsection{Intro}
\begin{frame}
\frametitle{Intro}
\begin{itemize}
\item A {\color{green}FIFO (First In First Out)} queue is where the item at the front of the queue is used first.
\item A new arrival joins the end.
\item All insertions are made at one end of the list.
\end{itemize}
\end{frame}
%-----------------------------------------------------------------

\begin{frame} 
\Huge{\centerline{The End}}
\end{frame}

\end{document}