%%%%%%%%%%%%%%%%%%%%%%%%%%%%%%%%%%%%%%%%%
% Beamer Presentation
% LaTeX Template
% Version 1.0 (10/11/12)
%
% This template has been downloaded from:
% http://www.LaTeXTemplates.com
%
% License:
% CC BY-NC-SA 3.0 (http://creativecommons.org/licenses/by-nc-sa/3.0/)
%
%%%%%%%%%%%%%%%%%%%%%%%%%%%%%%%%%%%%%%%%%

%----------------------------------------------------------------------------------------
%	PACKAGES AND THEMES
%----------------------------------------------------------------------------------------

\documentclass{beamer}

\mode<presentation> {

% The Beamer class comes with a number of default slide themes
% which change the colors and layouts of slides. Below this is a list
% of all the themes, uncomment each in turn to see what they look like.

%\usetheme{default}
%\usetheme{AnnArbor}
%\usetheme{Antibes}
%\usetheme{Bergen}
%\usetheme{Berkeley}
%\usetheme{Berlin}
%\usetheme{Boadilla}
%\usetheme{CambridgeUS}
%\usetheme{Copenhagen}
%\usetheme{Darmstadt}
%\usetheme{Dresden}
%\usetheme{Frankfurt}
%\usetheme{Goettingen}
%\usetheme{Hannover}
%\usetheme{Ilmenau}
%\usetheme{JuanLesPins}
%\usetheme{Luebeck}
\usetheme{Madrid}
%\usetheme{Malmoe}
%\usetheme{Marburg}
%\usetheme{Montpellier}
%\usetheme{PaloAlto}
%\usetheme{Pittsburgh}
%\usetheme{Rochester}
%\usetheme{Singapore}
%\usetheme{Szeged}
%\usetheme{Warsaw}

% As well as themes, the Beamer class has a number of color themes
% for any slide theme. Uncomment each of these in turn to see how it
% changes the colors of your current slide theme.

%\usecolortheme{albatross}
%\usecolortheme{beaver}
%\usecolortheme{beetle}
%\usecolortheme{crane}
%\usecolortheme{dolphin}
%\usecolortheme{dove}
%\usecolortheme{fly}
%\usecolortheme{lily}
%\usecolortheme{orchid}
%\usecolortheme{rose}
%\usecolortheme{seagull}
%\usecolortheme{seahorse}
%\usecolortheme{whale}
%\usecolortheme{wolverine}

%\setbeamertemplate{footline} % To remove the footer line in all slides uncomment this line
%\setbeamertemplate{footline}[page number] % To replace the footer line in all slides with a simple slide count uncomment this line

%\setbeamertemplate{navigation symbols}{} % To remove the navigation symbols from the bottom of all slides uncomment this line
}

\usepackage{graphicx} % Allows including images
\usepackage{booktabs} % Allows the use of \toprule, \midrule and \bottomrule in tables
\usepackage{listings}
\usepackage{amsmath}
\usepackage{algpseudocode,algorithm,algorithmicx}

\lstdefinestyle{custom}{
  breaklines=true,
  frame=L,
  xleftmargin=\parindent,
  language=Java,
  showstringspaces=false,
  basicstyle=\footnotesize\ttfamily,
  keywordstyle=\ttfamily\bfseries\color{green!40!black},
  commentstyle=\ttfamily\itshape\color{gray!40!black},
  identifierstyle=\color{blue},
  stringstyle=\color{orange},
  tabsize = 2
}

%----------------------------------------------------------------------------------------
%	TITLE PAGE
%----------------------------------------------------------------------------------------

\title[Merge Sort]{Merge Sort} % The short title appears at the bottom of every slide, the full title is only on the title page

\author{Jonathan Windle} % Your name
\institute[UEA] % Your institution as it will appear on the bottom of every slide, may be shorthand to save space
{
University of East Anglia \\ % Your institution for the title page
\medskip
\textit{J.Windle@uea.ac.uk} % Your email address
}
\date{\today} % Date, can be changed to a custom date

\begin{document}

\begin{frame}
\titlepage % Print the title page as the first slide
\end{frame}

\begin{frame}[allowframebreaks]
\frametitle{Overview} % Table of contents slide, comment this block out to remove it
\tableofcontents % Throughout your presentation, if you choose to use \section{} and \subsection{} commands, these will automatically be printed on this slide as an overview of your presentation
\end{frame} 

%------------------------------------------------------------------
\section{Intro}
\begin{frame}
\frametitle{Intro}
\begin{itemize}
\item It is a divide and conquer algorithm.
\item Partitions the existing array by rearranging the existing elements.
\item Partitions by chooding a pivot value, then swapping the elements so that all the elements to the left of the pivot are smaller than those on the right.
\end{itemize}
\begin{enumerate}
\item If the number of elements is size 0 or 1, then return.
\item Pick an element $v$ in $T$ as the $pivot$ element.
\item Partition $T$ without $v$ into two groups, the left group, $L$ consisting of elements smaller than or equal to $v$ and the right group $R$ consisting of elements greater than $v$.
\item $L$ = \texttt{quickSort($L$)} and $R$ = \texttt{quickSort($R$)}
\item return result of $L+pivot+R$.
\end{enumerate}
\end{frame}
%---------------------------------------------------------------
\defverbatim[colored]\quick{
\begin{lstlisting}[style = custom]
if(low == high)
	return T[low]
if(high > low)
	return null
pivot = choosePivot()
left = partitionLeft()
right = partitionRight()
left = quickSort(left)
right = quickSort(right)
full = left+pivot+right
\end{lstlisting}
}
\section{Algorithm}
\begin{frame}
\frametitle{Algorithm}
\quick
\end{frame}
%--------------------------------------------------------------
\section{Choosing a Pivot}
\begin{frame}
\frametitle{Choosing a Pivot}
\begin{itemize}
\item Idea pivot would split the array exactly in two
\item The middle value of set of numbers is the median
\item Finding the median takes time
\item Usual strategy is to take k elements randomly and then take the median of these.
\item Cmmonly used extension involves taking the medians of medians.
\end{itemize}
\end{frame}
%-----------------------------------------------------------------
\defverbatim[colored]\partit{
\begin{lstlisting}[style = custom]
temp = T[start]
T[start] = T[pivotPos]
T[pivotPos] = temp
left = start+1
right = end
while left <= right:
	while T[left]<=T[start]:
		left++
	while T[right]>T[start]:
		right++
	if left>right:
		temp = T[left]
		T[left] = T[right]
		T[right] = temp
	left++
	right--
temp = T[start]
T[start] = T[right]
T[right] = temp
\end{lstlisting}
}
\section{In place partitioning}
\begin{frame}[allowframebreaks]
\frametitle{In place partitioning}
\begin{enumerate}
\item Swap the pivot out of the way and set left and right to start+1.
\item Repeat until left $>$ right:
\begin{enumerate}
\item Advance the left pointer until the next element that should be in the right partition or end of array
\item Decrease the right pointer until the next element that should be in the left partition
\item Swap T[left] and T[right]
\item Add one to left and subtract one from right.
\end{enumerate}
\item Swap pivot with the last element in left partition: T[right]
\end{enumerate}
\partit
\end{frame}
%----------------------------------------------------------------
\section{Summary}
\begin{frame}
\frametitle{Summary}
\begin{itemize}
\item Quick sort is average case: $O(nlog)$ based on comparison.
\item Worst case is: $O(n^2)$ with standard pivoting methods.
\end{itemize}
\end{frame}
%----------------------------------------------------------------
\begin{frame} 
\Huge{\centerline{The End}}
\end{frame}

\end{document}