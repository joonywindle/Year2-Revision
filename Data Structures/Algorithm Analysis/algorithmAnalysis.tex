%%%%%%%%%%%%%%%%%%%%%%%%%%%%%%%%%%%%%%%%%
% Beamer Presentation
% LaTeX Template
% Version 1.0 (10/11/12)
%
% This template has been downloaded from:
% http://www.LaTeXTemplates.com
%
% License:
% CC BY-NC-SA 3.0 (http://creativecommons.org/licenses/by-nc-sa/3.0/)
%
%%%%%%%%%%%%%%%%%%%%%%%%%%%%%%%%%%%%%%%%%

%----------------------------------------------------------------------------------------
%	PACKAGES AND THEMES
%----------------------------------------------------------------------------------------

\documentclass{beamer}

\mode<presentation> {

% The Beamer class comes with a number of default slide themes
% which change the colors and layouts of slides. Below this is a list
% of all the themes, uncomment each in turn to see what they look like.

%\usetheme{default}
%\usetheme{AnnArbor}
%\usetheme{Antibes}
%\usetheme{Bergen}
%\usetheme{Berkeley}
%\usetheme{Berlin}
%\usetheme{Boadilla}
%\usetheme{CambridgeUS}
%\usetheme{Copenhagen}
%\usetheme{Darmstadt}
%\usetheme{Dresden}
%\usetheme{Frankfurt}
%\usetheme{Goettingen}
%\usetheme{Hannover}
%\usetheme{Ilmenau}
%\usetheme{JuanLesPins}
%\usetheme{Luebeck}
\usetheme{Madrid}
%\usetheme{Malmoe}
%\usetheme{Marburg}
%\usetheme{Montpellier}
%\usetheme{PaloAlto}
%\usetheme{Pittsburgh}
%\usetheme{Rochester}
%\usetheme{Singapore}
%\usetheme{Szeged}
%\usetheme{Warsaw}

% As well as themes, the Beamer class has a number of color themes
% for any slide theme. Uncomment each of these in turn to see how it
% changes the colors of your current slide theme.

%\usecolortheme{albatross}
%\usecolortheme{beaver}
%\usecolortheme{beetle}
%\usecolortheme{crane}
%\usecolortheme{dolphin}
%\usecolortheme{dove}
%\usecolortheme{fly}
%\usecolortheme{lily}
%\usecolortheme{orchid}
%\usecolortheme{rose}
%\usecolortheme{seagull}
%\usecolortheme{seahorse}
%\usecolortheme{whale}
%\usecolortheme{wolverine}

%\setbeamertemplate{footline} % To remove the footer line in all slides uncomment this line
%\setbeamertemplate{footline}[page number] % To replace the footer line in all slides with a simple slide count uncomment this line

%\setbeamertemplate{navigation symbols}{} % To remove the navigation symbols from the bottom of all slides uncomment this line
}

\usepackage{graphicx} % Allows including images
\usepackage{booktabs} % Allows the use of \toprule, \midrule and \bottomrule in tables
\usepackage{listings}
\usepackage{amsmath}
\usepackage{algpseudocode,algorithm,algorithmicx}


%----------------------------------------------------------------------------------------
%	TITLE PAGE
%----------------------------------------------------------------------------------------

\title[Algorithm Analysis]{Algorithm Analysis} % The short title appears at the bottom of every slide, the full title is only on the title page

\author{Jonathan Windle} % Your name
\institute[UEA] % Your institution as it will appear on the bottom of every slide, may be shorthand to save space
{
University of East Anglia \\ % Your institution for the title page
\medskip
\textit{J.Windle@uea.ac.uk} % Your email address
}
\date{\today} % Date, can be changed to a custom date

\begin{document}

\begin{frame}
\titlepage % Print the title page as the first slide
\end{frame}

\begin{frame}[allowframebreaks]
\frametitle{Overview} % Table of contents slide, comment this block out to remove it
\tableofcontents % Throughout your presentation, if you choose to use \section{} and \subsection{} commands, these will automatically be printed on this slide as an overview of your presentation
\end{frame}

%------------------------------------------------------------------
\section{Algorithm Design}
\subsection{Intro}
\begin{frame}
\frametitle{Intro}
\begin{itemize}
\item An algorithm is a step by step process for solving a problem. It consits of a finite sequence of instructions that when carried out, always terminates.
\item Algorithms manipulate data structures.
\item They are developed through a process of refinement, from an informal description to a formal description.
\item Formal description is written in pseudocode.
\end{itemize}
\end{frame}

%--------------------------------------------------------------------
\subsection{Developing Algorithms}
\begin{frame}
\frametitle{Developing Algorithms}
Steps to designing algorithm
\begin{enumerate}
\item Express in general terms how the algorithm works.
\item Give more detailed, but still informal description of the algorithm, identifying subproblems.
\item May be necessary to treat subproblems in the same way, known as {\color{green} step-wise refinement}.
\item Give detailed, unambiguous description in {\color{red}pseudo-code}.
\end{enumerate}
\end{frame}

%-----------------------------------------------------------------
\subsection{Algorithm Differences}
\begin{frame}
\frametitle{Algorithm Differences}
\begin{itemize}
\item Differeneces between algorithms can impact dramatically the speed of execution.
\begin{itemize}
\item This is measured by {\color{red} run-time efficiency}
\end{itemize}
\item Differences can also mean they have different memory requirements
\begin{itemize}
\item They are said to have different {\color{green} space efficiency}
\end{itemize}
\item There is often a trade-off between the two.
\end{itemize}
\end{frame}

%-------------------------------------------------------------------
\subsection{Recursion}
\begin{frame}
\frametitle{Recursion}
\begin{itemize}
\item A recursive definition is something that is defined in terms of itself. ({\color{green} A method calling itself}).
\item Example is definition of factorial n:
\begin{equation}
\begin{split}
1! & = 1 \\
n! & = n \times (n - 1)!,\quad for \quad n > 1
\end{split}
\end{equation}
\item Rules for a recursive algorithm:
\begin{enumerate}
\item Must have at least one {\color{red} base case} and one {\color{purple} recursive case}.
\item The {\color{purple}recursive case} should ensure that the {\color{red} base case} is eventually reached.
\end{enumerate}
\end{itemize}
\end{frame}
%------------------------------------------------------------------
\subsection{Tail Recursion}
\begin{frame}
\frametitle{Tail Recursion}
\begin{itemize}
\item An algorithm is {\color{green} tail recursive} if there is nothing to do after the return (except return its value).
\item For example, this return is \textbf{NOT} {\color{green} tail recursive} because it has to be multiplied by n before return:
\texttt{return n $\times$ factorial(n-1);}
\item This, however \textbf{IS} {\color{green} tail recursive}: \texttt{return gcd(y,x\%y);}
\item {\color{green} Tail recursive} algorithms can easily be turne into {\color{red} iterative} algorithms.
\item {\color{red} Iterative} is usually more efficient to use.
\end{itemize}
\end{frame}

%-----------------------------------------------------------------
\section{Algorithm Analysis}
\subsection{Run-time analysis}
\begin{frame}
\frametitle{Run-time analysis}
\begin{itemize}
\item Strategy:
\begin{enumerate}
\item Decide on {\color{green} fundamental operation}.
\item Decide on the {\color{red}case}.
\item Count the {\color{orange} number} of occurances given {\color{red}case} (Usually the {\color{magenta} worst-case}).
\item Form the {\color{purple} run-time complexity function} $t(n)$ where $n$ is the problem size.
\item Characterise the {\color{brown} order}.
\end{enumerate}
\end{itemize}
\end{frame}
%-------------------------------------------------------------------
\subsection{Characterising $t(n)$}
\begin{frame}
\frametitle{Characterising $t(n)$}
In order of best to worst:
\begin{tabular}{|c|c|}
\hline
Constant & $t(n) = c$ \\
\hline
Logarithmic & $t(n) = log n$\\
\hline
Linear & $t(n) = n$ \\
\hline
Log Linear & $t(n) = nlogn$ \\
\hline
Quadratic & $t(n) = n^2$\\
\hline
Polynomial Degree $p$ & $t(n) = n^p$\\
\hline
Exponential & $t(n) = 2^n$\\
\hline
\end{tabular}
\end{frame}
%--------------------------------------------------------------
\subsection{Useful Summations}
\begin{frame}
\frametitle{Useful Summations}
\begin{itemize}
\item $\displaystyle\sum_{i=1}^{n}( c) = c \times n$
\item $\displaystyle\sum_{i=1}^{n}(i) = \frac{1}{2}n(n+1)$
\item $\displaystyle\sum_{i=1}^{n}(c \times f(i)) = c \times \displaystyle\sum_{i=1}^{n}(f(i))$
\item $\displaystyle\sum_{i=1}^{n}(f(i) + g(i)) = \displaystyle\sum_{i=1}^{n}(f(i)) + \displaystyle\sum_{i=1}^{n}(g(i))$
\item $\displaystyle\sum_{i=m}^{n}(f(i)) = \displaystyle\sum_{i=1}^{n}(f(i)) - \displaystyle\sum_{i=1}^{m-1}(f(i))$
\end{itemize}
\end{frame}
%-----------------------------------------------------------------
\subsection{Notations}
\begin{frame}
\frametitle{Notations}
\begin{itemize}
\item $O(f(n)): Big O$
\begin{itemize}
\item For all $n$ sufficiently large, $t(n)$ grows at no greater rate than $f(n)$.
\item This gives the {\color{red} upper-bound} for the rate of growth.
\end{itemize}
\begin{itemize}
\item $\Omega(f(n)): Omega n$
\begin{itemize}
\item This denotes the {\color{green} lower-bound} for the rate of growth.
\end{itemize}
\item When an algorithm is both $\Omega(f(n))$ and $O(f(n))$ then it is said to be $\Theta(f(n))$.
\end{itemize}
\end{itemize}
\end{frame}
%----------------------------------------------------------------
\subsection{Analysis Example}
\begin{frame}
\frametitle{Analysis Example - Algorithm}
\begin{itemize}
\item A matrix is {\color{green} Upper-triangular} if all the elements below the diagonal are 0.
\begin{algorithm}[H]
  \caption{Output true if $M$ is upper-triangular, false otherwise}
  \begin{algorithmic}[1]
    \Require{N/A}
    \Statex
    \Function{isUpperTriangular}{$M,n$}
    	\For{$i \leftarrow 2 to n$}
    		\For{$j \leftarrow 1 to i-1$}
    			\If{$M[i,j] \neq 0$}
    				\State\Return{$false$}
    			\EndIf
    		\EndFor
    	\EndFor
    	\State \Return{$true$}
    \EndFunction
  \end{algorithmic}
\end{algorithm}
\end{itemize}
\end{frame}

%-----------------------------------------------------------------
\begin{frame}
\frametitle{Analysis Example - Analysis}
\fontsize{6pt}{7.2}
\begin{enumerate}
\item Determine fundamental operation:
\begin{itemize}
\tiny
\item Comparisoon: $M[i,j] \neq 0$
\end{itemize}
\item Decide on the case (Worst/Best)
\begin{itemize}
\tiny
\item worst-case, the worst case is that the matrix is upper-triangular.
\end{itemize}
\item Determine run-time complexity function $t(n)$ by counting fundamental operations
\begin{equation}
\tiny
\begin{split}
t(n) & = \displaystyle\sum_{i=2}^{n}\displaystyle\sum_{j = 1}^{i-1}(1)\\
& = \displaystyle\sum_{i=2}^{n}(i-1)\\
& = \displaystyle\sum_{i=2}^{n}(i) - \displaystyle\sum_{i=2}^{n}(1)\\
& = \displaystyle\sum_{i=1}^{n-1}(i+1) - (n-1) \\
& = \displaystyle\sum_{i=1}^{n-1}(i) + \displaystyle\sum_{i=1}^{n-1}(1) - (n-1) \\
& = \frac{1}{2}n(n-1)+(n-1)-(n-1)\\
& = \frac{1}{2}n(n-1)
\end{split}
\end{equation}
\item Give the order characterised by dominant term:
\begin{itemize}
\tiny
\item The worst case time complexity of this is both $O(n^2)$ and $\Omega(n^2)$ and is therefore $\Theta(n^2)$
\end{itemize}
\end{enumerate}
\end{frame}
%-----------------------------------------------------------------

\begin{frame} 
\Huge{\centerline{The End}}
\end{frame}

\end{document}