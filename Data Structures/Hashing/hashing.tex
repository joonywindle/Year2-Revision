%%%%%%%%%%%%%%%%%%%%%%%%%%%%%%%%%%%%%%%%%
% Beamer Presentation
% LaTeX Template
% Version 1.0 (10/11/12)
%
% This template has been downloaded from:
% http://www.LaTeXTemplates.com
%
% License:
% CC BY-NC-SA 3.0 (http://creativecommons.org/licenses/by-nc-sa/3.0/)
%
%%%%%%%%%%%%%%%%%%%%%%%%%%%%%%%%%%%%%%%%%

%----------------------------------------------------------------------------------------
%	PACKAGES AND THEMES
%----------------------------------------------------------------------------------------

\documentclass{beamer}

\mode<presentation> {

% The Beamer class comes with a number of default slide themes
% which change the colors and layouts of slides. Below this is a list
% of all the themes, uncomment each in turn to see what they look like.

%\usetheme{default}
%\usetheme{AnnArbor}
%\usetheme{Antibes}
%\usetheme{Bergen}
%\usetheme{Berkeley}
%\usetheme{Berlin}
%\usetheme{Boadilla}
%\usetheme{CambridgeUS}
%\usetheme{Copenhagen}
%\usetheme{Darmstadt}
%\usetheme{Dresden}
%\usetheme{Frankfurt}
%\usetheme{Goettingen}
%\usetheme{Hannover}
%\usetheme{Ilmenau}
%\usetheme{JuanLesPins}
%\usetheme{Luebeck}
\usetheme{Madrid}
%\usetheme{Malmoe}
%\usetheme{Marburg}
%\usetheme{Montpellier}
%\usetheme{PaloAlto}
%\usetheme{Pittsburgh}
%\usetheme{Rochester}
%\usetheme{Singapore}
%\usetheme{Szeged}
%\usetheme{Warsaw}

% As well as themes, the Beamer class has a number of color themes
% for any slide theme. Uncomment each of these in turn to see how it
% changes the colors of your current slide theme.

%\usecolortheme{albatross}
%\usecolortheme{beaver}
%\usecolortheme{beetle}
%\usecolortheme{crane}
%\usecolortheme{dolphin}
%\usecolortheme{dove}
%\usecolortheme{fly}
%\usecolortheme{lily}
%\usecolortheme{orchid}
%\usecolortheme{rose}
%\usecolortheme{seagull}
%\usecolortheme{seahorse}
%\usecolortheme{whale}
%\usecolortheme{wolverine}

%\setbeamertemplate{footline} % To remove the footer line in all slides uncomment this line
%\setbeamertemplate{footline}[page number] % To replace the footer line in all slides with a simple slide count uncomment this line

%\setbeamertemplate{navigation symbols}{} % To remove the navigation symbols from the bottom of all slides uncomment this line
}

\usepackage{graphicx} % Allows including images
\usepackage{booktabs} % Allows the use of \toprule, \midrule and \bottomrule in tables
\usepackage{listings}
\usepackage{amsmath}
\usepackage{algpseudocode,algorithm,algorithmicx}

\lstdefinestyle{custom}{
  breaklines=true,
  frame=L,
  xleftmargin=\parindent,
  language=Java,
  showstringspaces=false,
  basicstyle=\footnotesize\ttfamily,
  keywordstyle=\ttfamily\bfseries\color{green!40!black},
  commentstyle=\ttfamily\itshape\color{gray!40!black},
  identifierstyle=\color{blue},
  stringstyle=\color{orange},
  tabsize = 2
}

%----------------------------------------------------------------------------------------
%	TITLE PAGE
%----------------------------------------------------------------------------------------

\title[Hashing]{Hashing} % The short title appears at the bottom of every slide, the full title is only on the title page

\author{Jonathan Windle} % Your name
\institute[UEA] % Your institution as it will appear on the bottom of every slide, may be shorthand to save space
{
University of East Anglia \\ % Your institution for the title page
\medskip
\textit{J.Windle@uea.ac.uk} % Your email address
}
\date{\today} % Date, can be changed to a custom date

\begin{document}

\begin{frame}
\titlepage % Print the title page as the first slide
\end{frame}

\begin{frame}[allowframebreaks]
\frametitle{Overview} % Table of contents slide, comment this block out to remove it
\tableofcontents % Throughout your presentation, if you choose to use \section{} and \subsection{} commands, these will automatically be printed on this slide as an overview of your presentation
\end{frame} 

%------------------------------------------------------------------
\section{Intro}
\begin{frame}
\frametitle{Intro}
\small
\begin{itemize}
\item Technique for performing insertions,deletions and finds in a dictionary in {\color{green}constant average time}.
\item \textbf{Hash table:}
\begin{itemize}
\small
\item An array, $T$ of some fixed size is  used to store the keys.
\item \texttt{size} referres to the size of $T$.
\item $S = \{0,1,...,size -1\}$
\end{itemize}
\item \textbf{Hashing function:}
\begin{itemize}
\small
\item $h: K \rightarrow S.$
\item Suppose $K$ is the set of 6 digit non-negative integers, then a possible (but poor) choce for $h$ is:\\
\center
$h(k)=k(mod 1000)$
\end{itemize}
\item \textbf{Collisions:}
\begin{itemize}
\small
\item A collision occurs when two keys hash to the same location in the hash table:\\
$h(k) = h(k').$\\
\item Want to choose the hash function to minimise the chance of collisions.
\item Need to decide how to handle collisions when they do occur.
\end{itemize}
\end{itemize}
\end{frame}
%----------------------------------------------------------------
\section{Choosing Hash Function}
\begin{frame}
\frametitle{Choosing a Hash Function}
\begin{itemize}
\item A good hash function maps keys {\color{red}uniformly} and {\color{green}randomly} into the full range of possible locations.
\item A good hash function should depend on all of the charactersf the characters in a key, but this is not a sufficient condition for a good hash function.
\item Must not just depend on all of the characters in a key but must also distribute keys evenly over the table.
\item The built in Java function \texttt{hashCode} returns an integer based on the objects {\color{blue}reference} unless the object is a string then it is based on the string itself.
\item The Java class \texttt{HashTable} can be used with keys of any user-defined data type provided an instance method \texttt{hashCode} is defined.
\end{itemize}
\end{frame}
%----------------------------------------------------------------
\section{Resolving Collisions}
\begin{frame}
\frametitle{Resolving Collisions}
\begin{itemize}
\item Use some other location that is open in the table:
\begin{itemize}
\item {\color{red} Open addressing}
\end{itemize}
\item Change the structure of the hash table so that each location can correspond to more than one value:
\begin{itemize}
\item {\color{green}Chaining}.
\item {\color{purple}Buckets}.
\end{itemize}
\end{itemize}
\end{frame}
%----------------------------------------------------------------
\subsection{Chaining/Buckets}
\begin{frame}
\frametitle{Chaining/Buckets}
\begin{itemize}
\item Chaining:
\begin{itemize}
\item For each location $T$, keep a {\color{red} list} of allthe keys hashed to that location.
\item Each entry in $T$ is thus a reference to a linked list of keys.
\item To form a search, just hash to find the list and then perform the appropriate operation.
\end{itemize}
\item Buckets:
\begin{itemize}
\item Each location in the hash table is a bucket.
\item A fixed number, $b$ of locations to store the keys.
\item Total space available is thus:\\
$size \times b$
\end{itemize}
\end{itemize}
\end{frame}
%----------------------------------------------------------------
\subsection{Open Addressing}
\begin{frame}
\frametitle{Open Addressing}
\begin{itemize}
\item If a collision occurs, alternative cells in $T$ are tried until an empty cell is found.
\item Locating an open loaction in te hash table is called {\color{green}probing} 
\item May be necessary to try more than one alternqative location.
\item The locations examined when a new key is inserted is called a {\color{purple}probe sequence}.
\item Let $\langle S_j^k\rangle$ denote the probe sequence then:\\
$s_0^k = h(k)$\\
$s_j^k = (s^k_{j-1} + p(j,k))\%size, \quad j \geq 1$.\\
\item Where $p(j,k)$ is called a {\color{red}probe increment}.
\item In the simplist scheme the probe increment is independant of both $j$ and $k$. i.e. it is a constant $p$ in particular {\color{magenta}linear probing}, $p=1$.
\end{itemize}
\end{frame}
%-----------------------------------------------------------------

\begin{frame} 
\Huge{\centerline{The End}}
\end{frame}

\end{document}