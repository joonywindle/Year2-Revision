%%%%%%%%%%%%%%%%%%%%%%%%%%%%%%%%%%%%%%%%%
% Beamer Presentation
% LaTeX Template
% Version 1.0 (10/11/12)
%
% This template has been downloaded from:
% http://www.LaTeXTemplates.com
%
% License:
% CC BY-NC-SA 3.0 (http://creativecommons.org/licenses/by-nc-sa/3.0/)
%
%%%%%%%%%%%%%%%%%%%%%%%%%%%%%%%%%%%%%%%%%

%----------------------------------------------------------------------------------------
%	PACKAGES AND THEMES
%----------------------------------------------------------------------------------------

\documentclass{beamer}

\mode<presentation> {

% The Beamer class comes with a number of default slide themes
% which change the colors and layouts of slides. Below this is a list
% of all the themes, uncomment each in turn to see what they look like.

%\usetheme{default}
%\usetheme{AnnArbor}
%\usetheme{Antibes}
%\usetheme{Bergen}
%\usetheme{Berkeley}
%\usetheme{Berlin}
%\usetheme{Boadilla}
%\usetheme{CambridgeUS}
%\usetheme{Copenhagen}
%\usetheme{Darmstadt}
%\usetheme{Dresden}
%\usetheme{Frankfurt}
%\usetheme{Goettingen}
%\usetheme{Hannover}
%\usetheme{Ilmenau}
%\usetheme{JuanLesPins}
%\usetheme{Luebeck}
\usetheme{Madrid}
%\usetheme{Malmoe}
%\usetheme{Marburg}
%\usetheme{Montpellier}
%\usetheme{PaloAlto}
%\usetheme{Pittsburgh}
%\usetheme{Rochester}
%\usetheme{Singapore}
%\usetheme{Szeged}
%\usetheme{Warsaw}

% As well as themes, the Beamer class has a number of color themes
% for any slide theme. Uncomment each of these in turn to see how it
% changes the colors of your current slide theme.

%\usecolortheme{albatross}
%\usecolortheme{beaver}
%\usecolortheme{beetle}
%\usecolortheme{crane}
%\usecolortheme{dolphin}
%\usecolortheme{dove}
%\usecolortheme{fly}
%\usecolortheme{lily}
%\usecolortheme{orchid}
%\usecolortheme{rose}
%\usecolortheme{seagull}
%\usecolortheme{seahorse}
%\usecolortheme{whale}
%\usecolortheme{wolverine}

%\setbeamertemplate{footline} % To remove the footer line in all slides uncomment this line
%\setbeamertemplate{footline}[page number] % To replace the footer line in all slides with a simple slide count uncomment this line

%\setbeamertemplate{navigation symbols}{} % To remove the navigation symbols from the bottom of all slides uncomment this line
}

\usepackage{graphicx} % Allows including images
\usepackage{booktabs} % Allows the use of \toprule, \midrule and \bottomrule in tables
\usepackage{listings}
\usepackage{amsmath}
\usepackage{algpseudocode,algorithm,algorithmicx}

\lstdefinestyle{custom}{
  breaklines=true,
  frame=L,
  xleftmargin=\parindent,
  language=Java,
  showstringspaces=false,
  basicstyle=\footnotesize\ttfamily,
  keywordstyle=\ttfamily\bfseries\color{green!40!black},
  commentstyle=\ttfamily\itshape\color{gray!40!black},
  identifierstyle=\color{blue},
  stringstyle=\color{orange},
  tabsize = 2
}

%----------------------------------------------------------------------------------------
%	TITLE PAGE
%----------------------------------------------------------------------------------------

\title[Basic Sorting]{Basic Sorting} % The short title appears at the bottom of every slide, the full title is only on the title page

\author{Jonathan Windle} % Your name
\institute[UEA] % Your institution as it will appear on the bottom of every slide, may be shorthand to save space
{
University of East Anglia \\ % Your institution for the title page
\medskip
\textit{J.Windle@uea.ac.uk} % Your email address
}
\date{\today} % Date, can be changed to a custom date

\begin{document}

\begin{frame}
\titlepage % Print the title page as the first slide
\end{frame}

\begin{frame}[allowframebreaks]
\frametitle{Overview} % Table of contents slide, comment this block out to remove it
\tableofcontents % Throughout your presentation, if you choose to use \section{} and \subsection{} commands, these will automatically be printed on this slide as an overview of your presentation
\end{frame} 

%------------------------------------------------------------------
\section{Intro}
\begin{frame}
\frametitle{Intro}
\begin{itemize}
\item \textbf{Time Complexity:}
\begin{itemize}
\item worst/average/best cases
\item Order andruntime complexity function
\item Number of comparisons vs number of swaps
\end{itemize}
\item \textbf{Space Complexity:}
\begin{itemize}
\item Is it in {\color{red}place}, this requires just a constant amount of memory.
\end{itemize}
\item \textbf{Basis:}
\begin{itemize}
\item Comparison (selection, insertion, bubble, merge, quick and heap sort).
\item Grouping (counting, bucket and radix).
\end{itemize}
\item \textbf{Stability:}
\begin{itemize}
\item Algorithm stable if the ordering of equal items in the original array is maintained in the sorted array.
\end{itemize}
\end{itemize}
\end{frame}
%---------------------------------------------------------------------
\defverbatim[colored]\selec{
\begin{lstlisting}[style = custom]
for i = 1 to n-1:
	pos = i
	for j=i+1 to n:
		if T[j] < T[pos]
			pos = j
		endif
	endfor
	temp = T[pos]
	T[pos] = T[i]
	T[i] = temp
endfor
\end{lstlisting}
}
\section{Selection sort}
\begin{frame}
\frametitle{Selection sort}
\begin{enumerate}
\item Find the minimum value in the list.
\item Swap it with the value in the first position.
\item Repret the steps above for the for remainder of the list (starting in the second position).
\selec
\end{enumerate}
\end{frame}
%----------------------------------------------------------------
\subsection{Analysis}
\begin{frame}
\frametitle{Analysis}
\begin{itemize}
\item Fundamental operation: \texttt{T[j] < T[pos]}.
\item Worst case: All cases the same.
\item Complexity function:
\end{itemize}
\begin{equation}
\tiny
\begin{split}
t(n) & = \displaystyle\sum_{i=1}^{n-1}(\displaystyle\sum_{j=i+1}^n(1))\\
&= \displaystyle\sum_{i=1}^{n-1}(n-(i+1)+1)\\
&= \displaystyle\sum_{i=1}^{n-1}(n-i)\\
&= n(n-1)-\displaystyle\sum_{i=1}^{n-1}(i)\\
&= n(n-1)-\frac{n(n-1)}{2}\\
&= \frac{n(n-1)}{2}\\
&= \frac{n^2}{2} - \frac{n}{2}\\
&\Theta(n^2)
\end{split}
\end{equation}
\end{frame}
%------------------------------------------------------------------
\subsection{Summary}
\begin{frame}
\frametitle{Summary}
\begin{itemize}
\item \textbf{Basis:} Comparison
\item \textbf{Time Complexity:}
\begin{itemize}
\item Number of comparisons is $\Theta(n^2)$
\item Number of swaps is $\Theta(n)$.
\end{itemize}
\item \textbf{Space Complexity:} Constant (in place).
\item \textbf{Stability:} Not Stable
\end{itemize}
\end{frame}
%----------------------------------------------------------------
\defverbatim[colored]\bub{
\begin{lstlisting}[style = custom]
for i=n to 1:
	for j=1 to i-1:
		if T[j] > T[j+1]
			SWAP(T[j],T[j+1]
		endif
	endfor
endfor
\end{lstlisting}
}
\section{Bubble Sort}
\begin{frame}
\frametitle{Bubble sort}
\begin{enumerate}
\item Scan the array, swapping any out of order neighbouring elements.
\item Once the largest element is in the last position, the procedure is repeated to find the next largest, and so on.
\bub
\end{enumerate}
\end{frame}
%--------------------------------------------------------------
\subsection{Analysis}
\begin{frame}
\frametitle{Analysis}
\begin{itemize}
\item Fundamental operation: \texttt{T[j] > T[j+1]}.
\item Worst case: All cases are the same.
\item Complexity function:
\end{itemize}
\begin{equation}
\tiny
\begin{split}
t(n) & = \displaystyle\sum_{i=1}^{n}(\displaystyle\sum_{j=1}^{i-1}(1))\\
&= \displaystyle\sum_{i=1}^{n}(i-1)\\
&= \displaystyle\sum_{i=1}^{n}(i)-\displaystyle\sum_{i=1}^n(1)\\
&= \displaystyle\sum_{i=1}^{n}(i-n)\\
&= \frac{n(n-1)}{2}-n\\
&= \frac{n^2}{2} + \frac{n}{2} - n\\
&= \frac{n^2}{1} - \frac{n}{2}\\
&\Theta(n^2)
\end{split}
\end{equation}
\end{frame}
%--------------------------------------------------------------
\subsection{Summary}
\begin{frame}
\frametitle{Summary}
\begin{itemize}
\item \textbf{Basis:} Comparison
\item \textbf{Time Complexity:}
\begin{itemize}
\item Number of comparisons is $\Theta(n^2)$.
\item Number of swaps is $\Theta(n^2)$.
\end{itemize}
\item \textbf{Space Complexity:} Constant (in place sorting)
\item \textbf{Stability:} Stable
\end{itemize}
\end{frame}
%-----------------------------------------------------------------
\defverbatim[colored]\ins{
\begin{lstlisting}[style = custom]
for i=2 to n:
	temp = T[i]
	j = i-1
	while j > 0 and temp < T[j]
		T[j+1] = T[j]
		j = j-1
	endwhile
	T[j+1] = temp
endfor
\end{lstlisting}
}
\section{Insertion Sort}
\begin{frame}
\frametitle{Insertion Sort}
\begin{enumerate}
\item A list size 1 is already sorted.
\item Insert the element in position 2 into the already sorted list.
\item From position 3 to n repreat the steps above
\ins
\end{enumerate}
\end{frame}

%----------------------------------------------------------------
\subsection{Summary}
\begin{frame}
\frametitle{Summary}
\begin{itemize}
\item \textbf{Basis:} Comparison
\item \textbf{Time Complexity:}
\begin{itemize}
\item Worst Case and Average Case:
\begin{itemize}
\item Number of Comparisons: $O(n^2)$
\item Number of swaps: $O(n^2)$
\end{itemize}
\item Best case:
\begin{itemize}
\item O(n).
\end{itemize}
\end{itemize}
\item \textbf{Space Complexity:} Constant (in place)
\item \textbf{Stability:} Stable.
\end{itemize}
\end{frame}
%----------------------------------------------------------------
\begin{frame} 
\Huge{\centerline{The End}}
\end{frame}

\end{document}